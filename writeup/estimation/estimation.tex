\batchmode
\makeatletter
\def\input@path{{/Users/wesley/git/gwr/writeup/estimation//}}
\makeatother
\documentclass[english]{article}\usepackage[]{graphicx}\usepackage[]{color}
%% maxwidth is the original width if it is less than linewidth
%% otherwise use linewidth (to make sure the graphics do not exceed the margin)
\makeatletter
\def\maxwidth{ %
  \ifdim\Gin@nat@width>\linewidth
    \linewidth
  \else
    \Gin@nat@width
  \fi
}
\makeatother

\definecolor{fgcolor}{rgb}{0.345, 0.345, 0.345}
\newcommand{\hlnum}[1]{\textcolor[rgb]{0.686,0.059,0.569}{#1}}%
\newcommand{\hlstr}[1]{\textcolor[rgb]{0.192,0.494,0.8}{#1}}%
\newcommand{\hlcom}[1]{\textcolor[rgb]{0.678,0.584,0.686}{\textit{#1}}}%
\newcommand{\hlopt}[1]{\textcolor[rgb]{0,0,0}{#1}}%
\newcommand{\hlstd}[1]{\textcolor[rgb]{0.345,0.345,0.345}{#1}}%
\newcommand{\hlkwa}[1]{\textcolor[rgb]{0.161,0.373,0.58}{\textbf{#1}}}%
\newcommand{\hlkwb}[1]{\textcolor[rgb]{0.69,0.353,0.396}{#1}}%
\newcommand{\hlkwc}[1]{\textcolor[rgb]{0.333,0.667,0.333}{#1}}%
\newcommand{\hlkwd}[1]{\textcolor[rgb]{0.737,0.353,0.396}{\textbf{#1}}}%

\usepackage{framed}
\makeatletter
\newenvironment{kframe}{%
 \def\at@end@of@kframe{}%
 \ifinner\ifhmode%
  \def\at@end@of@kframe{\end{minipage}}%
  \begin{minipage}{\columnwidth}%
 \fi\fi%
 \def\FrameCommand##1{\hskip\@totalleftmargin \hskip-\fboxsep
 \colorbox{shadecolor}{##1}\hskip-\fboxsep
     % There is no \\@totalrightmargin, so:
     \hskip-\linewidth \hskip-\@totalleftmargin \hskip\columnwidth}%
 \MakeFramed {\advance\hsize-\width
   \@totalleftmargin\z@ \linewidth\hsize
   \@setminipage}}%
 {\par\unskip\endMakeFramed%
 \at@end@of@kframe}
\makeatother

\definecolor{shadecolor}{rgb}{.97, .97, .97}
\definecolor{messagecolor}{rgb}{0, 0, 0}
\definecolor{warningcolor}{rgb}{1, 0, 1}
\definecolor{errorcolor}{rgb}{1, 0, 0}
\newenvironment{knitrout}{}{} % an empty environment to be redefined in TeX

\usepackage{alltt}
\usepackage[T1]{fontenc}
\usepackage[latin9]{inputenc}
\setlength{\parskip}{\bigskipamount}
\setlength{\parindent}{0pt}
\usepackage{array}
\usepackage{verbatim}
\usepackage{bm}
\usepackage{multirow}
\usepackage{amsthm}
\usepackage{amsmath}
\usepackage{amssymb}
\usepackage{graphicx}
\usepackage{setspace}
\doublespacing

\makeatletter

%%%%%%%%%%%%%%%%%%%%%%%%%%%%%% LyX specific LaTeX commands.
%% Because html converters don't know tabularnewline
\providecommand{\tabularnewline}{\\}

%%%%%%%%%%%%%%%%%%%%%%%%%%%%%% Textclass specific LaTeX commands.
\usepackage[natbibapa]{apacite}
\theoremstyle{plain}
\newtheorem{thm}{\protect\theoremname}

\makeatother

\usepackage{babel}
\providecommand{\theoremname}{Theorem}
\IfFileExists{upquote.sty}{\usepackage{upquote}}{}
\begin{document}

\title{LAGR and its oracle properties}


\author{Wesley Brooks}

\maketitle

\section{Introduction}

\begin{comment}
Whereas the coefficients in traditional linear regression are scalar
constants, the coefficients in a varying coefficient regression (VCR)
model are functions - often \textbackslash{}emph\{smooth\} functions
- of some effect modifying variable \textbackslash{}citep\{Hastie:1993a\}. 

The methodology described herein is applicable to geostatistical data
and areal data. Let \$\textbackslash{}mathcal\{D\}\$ be a spatial
domain on which data is collected. For geostatistical data, let \$\textbackslash{}bm\{s\}\$
denote a location in \$\textbackslash{}mathcal\{D\}\$. Let a univariate
spatial process \$\textbackslash{}left\textbackslash{}\{Y(\textbackslash{}bm\{s\})
: \textbackslash{}bm\{s\} \textbackslash{}in \textbackslash{}mathcal\{D\}\textbackslash{}right\textbackslash{}\}\$
and a possibly multivariate spatial process \$\textbackslash{}left\textbackslash{}\{\textbackslash{}bm\{X\}(\textbackslash{}bm\{s\})
: \textbackslash{}bm\{s\} \textbackslash{}in \textbackslash{}mathcal\{D\}\textbackslash{}right\textbackslash{}\}\$
denote random fields of the response and the covariates, respectively.
For \$i = 1, \textbackslash{}dots, n\$, let \$\textbackslash{}bm\{s\}\_i\$
denote the sampling location in \$\textbackslash{}mathcal\{D\}\$ of
the \$i\$th observation of the response and the covariates. Let the
observed data be denoted \$\textbackslash{}left\textbackslash{}\{y(\textbackslash{}bm\{s\}\_i),
\textbackslash{}bm\{x\}(\textbackslash{}bm\{s\}\_i)\textbackslash{}right\textbackslash{}\}\$,
\$i=1, \textbackslash{}dots, n\$. Then the data are a realization
of the random fields at the sampling locations \$\textbackslash{}left\textbackslash{}\{Y(\textbackslash{}bm\{s\}\_i),
\textbackslash{}bm\{X\}(\textbackslash{}bm\{s\}\_i)\textbackslash{}right\textbackslash{}\}\$
for \$i=1, \textbackslash{}dots, n\$. 

For areal data, the spatial domain \$\textbackslash{}mathcal\{D\}\$
is partitioned into \$n\$ regions \$\textbackslash{}\{D\_1, \textbackslash{}dots,
D\_n\textbackslash{}\}\$ such that \$\textbackslash{}mathcal\{D\}
= \textbackslash{}bigcup \textbackslash{}limits\_\{i=1\}\textasciicircum{}n
D\_i\$. In the case of areal data, the random variables \$\textbackslash{}left\textbackslash{}\{Y(D\_i),
\textbackslash{}bm\{X\}(D\_i)\textbackslash{}right\textbackslash{}\}\$
are defined for regions instead of for point locations; population
and spatial mean temperature are examples of areal data. The analytical
method described herein can be applied to areal data if they are recast
as geostatistical data by assuming that the data are point-referenced
to a representative location of each region, such as the centroid.
That is, \$\textbackslash{}left\textbackslash{}\{\textbackslash{}bm\{X\}(\textbackslash{}bm\{s\}\_i),
Y(\textbackslash{}bm\{s\}\_i)\textbackslash{}right\textbackslash{}\}
\$ where \$\textbackslash{}bm\{s\}\_i\$ is the centroid of \$D\_i\$
for \$i=1, \textbackslash{}dots, n\$. 

Common practice in the analysis of geostatistical and areal data is
to model the response variable with a spatial linear regression model
consisting of the sum of a fixed mean function, a spatial random effect,
and random error all on domain \$\textbackslash{}mathcal\{D\}\$, as
in: \textbackslash{}begin\{align\}\textbackslash{}label\{eq:spatial-regression\}
Y(\textbackslash{}bm\{s\}) = \textbackslash{}bm\{X\}(\textbackslash{}bm\{s\})'\textbackslash{}bm\{\textbackslash{}beta\}
+ W(\textbackslash{}bm\{s\}) + \textbackslash{}varepsilon(\textbackslash{}bm\{s\})
\textbackslash{}end\{align\} where \$\textbackslash{}bm\{X\}(\textbackslash{}bm\{s\})'\textbackslash{}bm\{\textbackslash{}beta\}\$
is the mean function consisting of a vector of covariates \$\textbackslash{}bm\{X\}(\textbackslash{}bm\{s\})\$,
and a vector of regression coefficients \$\textbackslash{}bm\{\textbackslash{}beta\}\$.
The random error \$\textbackslash{}varepsilon(\textbackslash{}bm\{s\})\$
denotes white noise such that the errors are independent and identically
distributed with mean zero and variance \$\textbackslash{}sigma\textasciicircum{}2\$,
while the random component \$W(\textbackslash{}bm\{s\})\$ denotes
a mean-zero, second-order stationary random field that is independent
of the random error. The mean function captures the large-scale systematic
trend of the response, the spatial random field \$W(\textbackslash{}bm\{s\})\$
can be thought of as a small-scale spatial random effect, and the
error term \$\textbackslash{}varepsilon(\textbackslash{}bm\{s\})\$
captures micro-scale variation \textbackslash{}citep\{Cressie:1993\}.

It is common to pre-specify the form of a covariance function for
the spatial random effect \$W(\textbackslash{}bm\{s\})\$ \textbackslash{}citep\{Diggle:2007\}.
For example, the exponential covariance function (a special case of
the Mat\textbackslash{}'\{e\}rn class of covariance functions) has
the form \textbackslash{}begin\{align\}\textbackslash{}label\{eq:exponential-covariance\}
\textbackslash{}text\{Cov\}(W(\textbackslash{}bm\{s\}), W(\textbackslash{}bm\{t\}))
= \textbackslash{}sigma\textasciicircum{}2 \textbackslash{}exp\textbackslash{}left\textbackslash{}\{-\textbackslash{}phi\textasciicircum{}\{-1\}
\textbackslash{}delta(\textbackslash{}bm\{s\}, \textbackslash{}bm\{t\})
\textbackslash{}right\textbackslash{}\} \textbackslash{}end\{align\}
where \$\textbackslash{}sigma\textasciicircum{}2\$ is a variance parameter,
\$\textbackslash{}phi\$ is a range parameter, and \$\textbackslash{}delta(\textbackslash{}bm\{s\},
\textbackslash{}bm\{t\})\$ is the Euclidean distance between locations
\$\textbackslash{}bm\{s\}\$ and \$\textbackslash{}bm\{t\}\$. The general
form of a covariance function in the Mat\textbackslash{}'\{e\}rn class
is \textbackslash{}begin\{align\}\textbackslash{}label\{eq:matern-covarinace\}
\textbackslash{}text\{Cov\}(W(\textbackslash{}bm\{s\}), W(\textbackslash{}bm\{t\}))
= \textbackslash{}left\textbackslash{}\{\textbackslash{}Gamma(\textbackslash{}nu)
2\textasciicircum{}\{\textbackslash{}nu-1\} \textbackslash{}right\textbackslash{}\}\textasciicircum{}\{-1\}
\textbackslash{}left\textbackslash{}\{\textbackslash{}delta(\textbackslash{}bm\{s\},
\textbackslash{}bm\{t\}) \textbackslash{}phi\textasciicircum{}\{-1\}\textbackslash{}sqrt\{2\textbackslash{}nu\}\textbackslash{}right\textbackslash{}\}\textasciicircum{}\textbackslash{}nu
K\_\{\textbackslash{}nu\} \textbackslash{}left(\textbackslash{}delta(\textbackslash{}bm\{s\},
\textbackslash{}bm\{t\}) \textbackslash{}phi\textasciicircum{}\{-1\}\textbackslash{}sqrt\{2\textbackslash{}nu\}\textbackslash{}right)
\textbackslash{}end\{align\} where \$\textbackslash{}nu\$ denotes
the degree of smoothness, \$K\_\{\textbackslash{}nu\}\$ denotes the
modified Bessel equation of the second kind, and as before \$\textbackslash{}phi\$
denotes a range parameter and \$\textbackslash{}delta(\textbackslash{}bm\{s\},
\textbackslash{}bm\{t\})\$ the Euclidean distance between locations
\$\textbackslash{}bm\{s\}\$ and \$\textbackslash{}bm\{t\}\$. The exponential
covariance function corresponds to a Mat\textbackslash{}'\{e\}rn class
covariance function with \$\textbackslash{}nu = 1/2\$.

\%SVCR - justification \%Stationarity in spatial linear regression
A random field is said to be stationary if the joint distribution
of a the response at a finite set of locations does not change when
the set of locations are all shifted in space by a fixed spatial lag.
That is, letting \$\textbackslash{}left\textbackslash{}\{T(\textbackslash{}bm\{s\})
: \textbackslash{}bm\{s\} \textbackslash{}in \textbackslash{}mathcal\{D\}\textbackslash{}right\textbackslash{}\}\$
be a random field on spatial domain \$\textbackslash{}mathcal\{D\}\$
that takes value \$T(\textbackslash{}bm\{s\}\_i)\$ at location \$\textbackslash{}bm\{s\}\_i
\textbackslash{}in \textbackslash{}mathcal\{D\}\$ for \$i = 1, \textbackslash{}dots,
n\$, the random field \$T(\textbackslash{}bm\{s\})\$ is stationary
if \$F\_n\textbackslash{}left(T(\textbackslash{}bm\{s\}\_1), \textbackslash{}dots,
T(\textbackslash{}bm\{s\}\_n)\textbackslash{}right) = F\_n\textbackslash{}left(T(\textbackslash{}bm\{s\}\_1+\textbackslash{}bm\{h\}),
\textbackslash{}dots, T(\textbackslash{}bm\{s\}\_n+\textbackslash{}bm\{h\})\textbackslash{}right)\$
where \$F\_n(\textbackslash{}cdot)\$ is the joint distribution of
a length \$n\$ sample from \$T(\textbackslash{}bm\{s\})\$ and \$\textbackslash{}bm\{h\}\$
is a fixed spatial lag. The random field \$\textbackslash{}left\textbackslash{}\{T(\textbackslash{}bm\{s\})
: \textbackslash{}bm\{s\} \textbackslash{}in \textbackslash{}mathcal\{D\}\textbackslash{}right\textbackslash{}\}\$
is second-order stationary if the following are satisfied: \textbackslash{}begin\{align\}
E\textbackslash{}left\textbackslash{}\{ T(\textbackslash{}bm\{s\})
\textbackslash{}right\textbackslash{}\} \&= \textbackslash{}mu \textbackslash{}text\{
for all \} \textbackslash{}bm\{s\} \textbackslash{}in \textbackslash{}mathcal\{D\}
\textbackslash{}notag \textbackslash{}\textbackslash{} \textbackslash{}text\{var\}\textbackslash{}left\textbackslash{}\{T(\textbackslash{}bm\{s\})\textbackslash{}right\textbackslash{}\}
\&= \textbackslash{}sigma\textasciicircum{}2 < \textbackslash{}infty
\textbackslash{}text\{ for all \} \textbackslash{}bm\{s\} \textbackslash{}in
\textbackslash{}mathcal\{D\} \textbackslash{}notag \textbackslash{}\textbackslash{}
\textbackslash{}text\{cov\}\textbackslash{}left\textbackslash{}\{
T(\textbackslash{}bm\{s\}), T(\textbackslash{}bm\{s\} + \textbackslash{}bm\{h\})
\textbackslash{}right\textbackslash{}\} \&= C(\textbackslash{}bm\{h\})
\textbackslash{}end\{align\} where the function \$C(\textbackslash{}cdot)\$
depends only on the spatial lag \$\textbackslash{}bm\{h\}\$ and not
on the location \$\textbackslash{}bm\{s\}\$. \%the joint distribution
at any two locations in the domain does not change when the locations
are shifted by a fixed spatial lag. 

The coefficient vector \$\textbackslash{}bm\{\textbackslash{}beta\}\$
in (\textbackslash{}ref\{eq:spatial-regression\}) is a fixed constant.
The model can be made more flexible if the coefficients are described
by a stationary random field. Such a model is written \textbackslash{}begin\{align\}\textbackslash{}label\{eq:SVCR-process\}
Y(\textbackslash{}bm\{s\}) = \textbackslash{}bm\{X\}(\textbackslash{}bm\{s\})'\textbackslash{}bm\{\textbackslash{}beta\}(\textbackslash{}bm\{s\})
+ \textbackslash{}varepsilon(\textbackslash{}bm\{s\}) \textbackslash{}end\{align\}
where \$\textbackslash{}bm\{\textbackslash{}beta\}(\textbackslash{}bm\{s\})\$
is a random coefficient field with a Mat\textbackslash{}'\{e\}rn-class
covariance function and the spatial random effect \$W(\textbackslash{}bm\{s\})\$
included in the intercept \$\textbackslash{}beta\_0(\textbackslash{}bm\{s\})\$.
The random coefficient field \$\textbackslash{}bm\{\textbackslash{}beta\}(\textbackslash{}bm\{s\})\$
can be estimated by Markov Chain Monte Carlo (MCMC) methods under
the assumption that \$\textbackslash{}bm\{\textbackslash{}beta\}(\textbackslash{}bm\{s\})\$
is stationary \textbackslash{}citep\{Gelfand:2003\}.

\%The spatial random effect describes the spatial pattern in the deviations
from the systematic part of the model. When fitting the spatial regression
model (\textbackslash{}ref\{eq:spatial-regression\}), it is usually
required that the the fitted values of the spatial random effect and
of the residuals sum to zero, i.e. \$\textbackslash{}sum\textbackslash{}limits\_\{i=1\}\textasciicircum{}n\textbackslash{}hat\{W\}(\textbackslash{}bm\{s\}\_i)
= 0\$ and \$\textbackslash{}sum\textbackslash{}limits\_\{i=1\}\textasciicircum{}n\textbackslash{}hat\{\textbackslash{}varepsilon\}(\textbackslash{}bm\{s\}\_i)
= 0\$. This mode of analysis is appropriate when the systematic part
of the regression model does not vary between locations. On the other
hand, a VCR model is appropriate for the case where the systematic
part of the regression model does vary across locations.

\%Spatial VCR Alternatively, kernel-based and spline-based methods
can be considered for fitting VCR models without assuming the coefficients
are described by a stationary random field. 

Coefficients for a spline-based VCR model are estimated by maximizing
a penalized global likelihood, with the penalty calculated from the
wiggliness of the coefficient surface \textbackslash{}citep\{Wood:2006\}.
This contrasts to kernel-based estimates of the coefficients in a
VCR model, which maximize a local likelihood to estimate the local
coefficients at each sampling location \textbackslash{}citep\{Loader:1999\}.
\textbackslash{}cite\{Fan:1999\} demonstrated that the optimal kernel
bandwidth estimate for a VCR model can be found via a two-step technique. 

Model selection in VCR models may be local or global. Global selection
means including or excluding variables everywhere in the spatial domain,
while local selection means including or excluding variables at individual
locations within the spatial domain. For global model selection in
spline-based VCR models, \textbackslash{}cite\{Wang:2008a\} proposed
a SCAD penalty \textbackslash{}citep\{Fan:2001\} for variable selection
in spline-based VCR models with a univariate effect-modifying variable.
\textbackslash{}cite\{Antoniadis:2012a\} used the nonnegative Garrote
penalty \textbackslash{}citep\{Breiman:1995\} in P-spline-based VCR
models having a univariate effect-modifying variable. 

Wavelet methods for fitting SVCR models were explored by \textbackslash{}cite\{Shang-2011\}
and \textbackslash{}cite\{Zhang-2011\}. Sparsity in the wavelet coefficients
is achieved either by \$\textbackslash{}ell\_1\$-penalization (also
known as the Lasso \textbackslash{}citep\{Tibshirani:1996\}) \textbackslash{}citep\{Shang-2011\}
or by Bayesian variable selection \textbackslash{}citep\{Zhang-2011\}.
Sparsity in the wavelet domain does not imply sparsity in the covariates,
though, so neither method is suitable for local variable selection.

Geographically weighted regression (GWR) is a kernel-based method
for estimating the coefficients of an SVCR model where the kernel
weights are based on the distance between sampling locations \textbackslash{}citep\{Brundson:1998a,
Fotheringham:2002\}. At each sampling location, traditional GWR estimates
the local regression coefficients by the local likelihood \textbackslash{}citep\{Loader:1999\}.
As a kernel-based smoother for regression coefficients, traditional
GWR tends to exhibit bias near the boundary of the region being modeled
\textbackslash{}citep\{Hastie:1993b\}. One way to reduce the boundary-effect
bias is to model the coefficient surface as locally linear rather
than locally constant by including coefficient-by-location interactions
\textbackslash{}citep\{Wang:2008b\}. 

Current practice for VCR models relies on \textit{a priori} global
model selection to decide which variables should be included in the
model. The idea of using Lasso regularization for local variable selection
in a VCR model has appeared in the literature as the geographically
weighted Lasso (GWL) \citet{Wheeler-2009}. The GWL applies the Lasso
for local variable selection and uses a jackknife criterion for selection
of the Lasso tuning parameters. However, there are several drawbacks
to the GWL. First, the GWL is a lasso procedure, but the lasso does
not generally produce consistent estimates of the relevant covariates
\citet{Leng-2006}. Because the jackknife criterion can only be computed
at sampling locations where the response variable is observed, the
GWL cannot be used to impute missing values of the response variable
nor to interpolate the coefficient surface and/or the response variable
between sampling locations. 

The adaptive Lasso (AL) \textbackslash{}citep\{Zou:2006\} is an improvement
to the Lasso that does produce consistent estimates of the coefficients
and has been shown to have appealing properties for automating variable
selection, which under suitable conditions include the ``oracle\textquotedbl{}
property of asymptotically selecting exactly the correct set of covariates
for inclusion in a regression model. 

The remainder of this document is organized as follows. In Section
\textbackslash{}ref\{section:simulation\}, a simulation study is conducted
to assess the performance of the GWEN in variable selection and coefficient
estimation. An application to real data is presented in Section \textbackslash{}ref\{section:data-analysis\}. 
\end{comment}



\section{Varying coefficients regression\label{sec:vcr}}


\subsection{Model}

Consider $n$ data points, observed at sampling locations $\bm{s}_{i}=(s_{i,1}\;\; s_{i,2})^{T}$
for $i=1,\dots,\bm{s}_{n}$, which are distributed in a spatial domain
$D\subset\mathbb{R}^{2}$ according to a density $f(\bm{s})$. For
$i=1,\dots,n$, let $y(\bm{s}_{i})$ and $\bm{x}(\bm{s}_{i})$ denote,
respectively, the univariate response and the $(p+1)$-variate vector
of covariates measured at location $\bm{s}_{i}$. At each location
$\bm{s}_{i}$, assume that the outcome is related to the covariates
by a linear model where the coefficients $\bm{\beta}(\bm{s}_{i})$
may be spatially-varying and $\varepsilon(\bm{s}_{i})$ is random
error at location $\bm{s}_{i}$. That is, 
\begin{align}
y(\bm{s}_{i})=\bm{x}(\bm{s}_{i})'\bm{\beta}(\bm{s}_{i})+\varepsilon(\bm{s}_{i}).\label{eq:lm(s)}
\end{align}


Further assume that the error term $\varepsilon(\bm{s}_{i})$ is normally
distributed with zero mean and variance $\sigma^{2}$, and that $\varepsilon(\bm{s}_{i})$,
$i=1,\dots,n$ are independent. That is, 
\begin{align}
\bm{\varepsilon}\overset{iid}{\sim}\mathcal{N}\left(0,\sigma^{2}\right).\label{eq:err}
\end{align}



\subsection{Augment the covariates and the coefficients with location interactions}

In the context of nonparametric regression, the boundary-effect bias
can be reduced by local polynomial modeling, usually in the form of
a locally linear model \citet{Fan-Gijbels-1996}. Here, locally linear
coefficients are estimated by augmenting the local design matrix with
covariate-by-location interactions in two dimensions as proposed by
\citet{Wang-2008b}. The augmented local design matrix at location
$\bm{s}_{i}$ is 
\begin{align}
\bm{Z}(\bm{s}_{i})=\left(\bm{X}\:\: L_{i}\bm{X}\:\: M_{i}\bm{X}\right)
\end{align}


where $\bm{X}$ is the unaugmented matrix of covariates, $L_{i}=\text{diag}\{s_{i'_{1}}-s_{i_{1}}\}$
and $M_{i}=\text{diag}\{s_{i'_{2}}-s_{i_{2}}\}$ for $i'=1,\dots,n$.

Now we have that $Y(\bm{s}_{i})=\left\{ \bm{Z}(\bm{s}_{i})\right\} _{i}^{T}\bm{\zeta}(\bm{s}_{i})+\varepsilon(\bm{s}_{i})$,
where $\left\{ \bm{Z}(\bm{s}_{i})\right\} _{i}^{T}$ is the $i$th
row of the matrix $\bm{Z}(\bm{s}_{i})$ as a row vector, and $\bm{\zeta}(\bm{s}_{i})$
is the vector of local coefficients at location $\bm{s}_{i}$, augmented
with the local gradients of the coefficient surfaces in the two spatial
dimensions, indicated by $\nabla_{u}$ and $\nabla_{v}$:

\[
\bm{\zeta}(\bm{s}_{i})=\left(\bm{\beta}(\bm{s}_{i})^{T}\;\;\nabla_{u}\bm{\beta}(\bm{s}_{i})^{T}\;\;\nabla_{v}\bm{\beta}(\bm{s}_{i})^{T}\right)^{T}
\]



\subsection{Local likelihood}

The total log-likelihood of the observed data is the sum of the log-likelihood
of each individual observation: 
\begin{align}
\ell\left\{ \bm{\zeta}\right\} =-(1/2)\sum_{i=1}^{n}\left[\log{\sigma^{2}}+\sigma^{-2}\left\{ y(\bm{s}_{i})-\bm{z}'(\bm{s}_{i})\bm{\zeta}(\bm{s}_{i})\right\} ^{2}\right].\label{eq:coefficients}
\end{align}


Since there are a total of $n\times3(p+1)+1$ parameters for $n$
observations, the model is not identifiable and it is not possible
to directly maximize the total likelihood. But since the coefficient
functions are smooth, the coefficients at location $\bm{s}$ can approximate
the coefficients within some neighborhood of $\bm{s}$, with the quality
of the approximation declining as the distance from $\bm{s}$ increases.

This intuition is formalized by the local likelihood, which is maximized
at location $\bm{s}$ to estimate the local coefficients $\bm{\zeta}(\bm{s})$:
\begin{align}
\mathcal{L}\left\{ \bm{\zeta}(\bm{s})\right\}  & =\prod_{i=1}^{n}\left\{ \left(2\pi\sigma^{2}\right)^{-1/2}\exp\left[-(1/2)\sigma^{-2}\left\{ y(\bm{s}_{i})-\bm{z}'(\bm{s}_{i})\bm{\zeta}(\bm{s})\right\} ^{2}\right]\right\} ^{K_{h}(\|\bm{s}-\bm{s}_{i}\|)},\label{eq:local-likelihood}
\end{align}


The weights are computed from a kernel function $K_{h}(\cdot)$ such
as the Epanechnikov kernel: 
\begin{align}
K_{h}(\|\bm{s}_{i}-\bm{s}_{i'}\|) & =h^{-2}K\left(h^{-1}\|\bm{s}_{i}-\bm{s}_{i'}\|\right)\notag\label{eq:epanechnikov}\\
K(x) & =\begin{cases}
(3/4)(1-x^{2}) & \mbox{ if }x<1,\\
0 & \mbox{ if }x\geq1.
\end{cases}
\end{align}


Thus, the local log-likelihood function is, up to an additive constant:
\begin{align}
\ell\left\{ \bm{\zeta}(\bm{s})\right\}  & =-(1/2)\sum_{i=1}^{n}K_{h}(\|\bm{s}-\bm{s}_{i}\|)\left[\log{\sigma^{2}}+\sigma^{-2}\left\{ y(\bm{s}_{i})-\bm{z}'(\bm{s}_{i})\bm{\zeta}(\bm{s})\right\} ^{2}\right].\label{eq:local-log-likelihood}
\end{align}



\subsection{Estimating the coefficients}

Letting $\bm{W}(\bm{s})$ be a diagonal weight matrix where $W_{ii}(\bm{s})=K_{h}(\|\bm{s}-\bm{s}_{i}\|)$,
the local likelihood is maximized by weighted least squares: 
\begin{align}
\mathcal{S}\left\{ \bm{\zeta}(\bm{s})\right\}  & =(1/2)\left\{ \bm{Y}-\bm{Z}(\bm{s})\bm{\zeta}(\bm{s})\right\} ^{T}\bm{W}(\bm{s})\left\{ \bm{Y}-\bm{Z}(\bm{s})\bm{\zeta}(\bm{s})\right\} ^{T}\notag\label{eq:zeta-hat}\\
\therefore\tilde{\bm{\zeta}}(\bm{s}) & =\left\{ \bm{Z}^{T}(\bm{s})\bm{W}(\bm{s})\bm{Z}(\bm{s})\right\} ^{-1}\bm{Z}^{T}(\bm{s})\bm{W}(\bm{s})\bm{Y}
\end{align}


Now Theorem 3 of \citet{Sun-Yan-Zhang-Lu-2014} says that, for any
given $\bm{{s}}$

\[
\sqrt{{nh^{2}f(\bm{{s}})}}\left[\hat{\bm{\beta}}(\bm{s})-\bm{\beta}(\bm{s})-(1/2)\kappa_{0}^{-1}\kappa_{2}h^{2}\left\{ \bm{\beta}_{uu}(\bm{s})+\bm{\beta}_{vv}(\bm{s})\right\} \right]\xrightarrow{{D}}N\left(\bm{0},\kappa_{0}^{-2}\nu_{0}\sigma^{2}\Psi^{-1}\right)
\]



\section{Local variable selection with LAGR\label{sec:lagr-gaussian}}

Estimating the local coefficients by (\ref{eq:zeta-hat}) relies on
\emph{a priori} variable selection. The goal of local adaptive grouped
regularization (LAGR) is to simultaneously select the locally relevant
predictors and estimate the local coefficients. The proposed LAGR
penalty is an adaptive $\ell_{1}$ penalty akin to the adaptive group
lasso \citep{Wang-Leng-2008,Zou-2006}.


\subsection{Variable groupings}

Each raw covariate in a LAGR model is grouped with its covariate-by-location
interactions. That is, $\bm{\zeta}_{j}(\bm{s})=\left(\beta_{j}(\bm{s})\;\;\;\nabla_{u}\beta_{j}(\bm{s})\;\;\;\nabla_{v}\beta_{j}(\bm{s})\right)^{T}$
for $j=1,\dots,p$. By the mechanism of the group lasso, variables
within the same group are included in or dropped from the model together.
The intercept group is left unpenalized.


\subsection{The LAGR-penalized local likelihood}

The objective function for the LAGR at location $\bm{s}$ is the penalized
local sum of squares: 
\begin{align}
Q\{\bm{\zeta}(\bm{s})\} & =\mathcal{S}\left\{ \bm{\zeta}(\bm{s})\right\} +\mathcal{J}\{\bm{\zeta}(\bm{s})\}\notag\label{eq:adaptive-lasso-WLS}\\
 & =(1/2)\left\{ \bm{Y}-\bm{Z}(\bm{s})\bm{\zeta}(\bm{s})\right\} ^{T}\bm{W}(\bm{s})\left\{ \bm{Y}-\bm{Z}(\bm{s})\bm{\zeta}(\bm{s})\right\} ^{T}+\sum_{j=1}^{p}\phi_{j}(\bm{s})\|\bm{\zeta}_{j}(\bm{s})\|
\end{align}


which is the sum of the weighted sum of squares $\mathcal{S}\left\{ \bm{\zeta}(\bm{s})\right\} $
and the LAGR penalty $\mathcal{J}\{\bm{\zeta}(\bm{s})\}$.

The LAGR penalty for the $j$th group of coefficients $\bm{\zeta}_{j}(\bm{s})$
at location $\bm{s}$ is $\phi_{j}(\bm{s})=\lambda_{n}(\bm{s})\|\tilde{\bm{\zeta}}_{j}(\bm{s})\|^{-\gamma}$,
where $\lambda_{n}(\bm{s})>0$ is a the local tuning parameter applied
to all coefficients at location $\bm{s}$ and $\tilde{\bm{\zeta}}_{j}(\bm{s})$
is the vector of unpenalized local coefficients from (\ref{eq:zeta-hat}).


\subsection{Oracle properties of LAGR}
\begin{thm}[Asymptotic normality]
\label{theorem:normality} 



If $h^{-1}n^{-1/2}a_{n}\xrightarrow{p}0$ and $hn^{-1/2}b_{n}\xrightarrow{p}\infty$
then 
\[
h\sqrt{n}\left[\hat{\bm{\beta}}_{(a)}(\bm{s})-\bm{\beta}_{(a)}(\bm{s})-\frac{\kappa_{2}h^{2}}{2\kappa_{0}}\{\nabla_{uu}^{2}\bm{\beta}_{(a)}(\bm{s})+\nabla_{vv}^{2}\bm{\beta}_{(a)}(\bm{s})\}\right]\xrightarrow{d}N(0,f(\bm{s})^{-1}\kappa_{0}^{-2}\nu_{0}\sigma^{2}\Psi^{-1})
\]

\end{thm}

\begin{thm}[Selection consistency]
\label{theorem:selection}



If $h^{-1}n^{-1/2}a_{n}\xrightarrow{p}\infty$ and $hn^{-1/2}b_{n}\xrightarrow{p}\infty$
then $P\left\{ \|\hat{\bm{\zeta}}_{j}(\bm{s})\|=0\right\} \to0$ if
$j\le p_{0}$ and $P\left\{ \|\hat{\bm{\zeta}}_{j}(\bm{s})\|=0\right\} \to1$
if $j>p_{0}$. 
\end{thm}

\paragraph{Remarks}

Together, Theorem�\ref{theorem:normality} and Theorem \ref{theorem:selection}
indicate that the LAGR estimates have the same asymptotic distribution
as a local regression model where the nonzero coefficients are known
in advance \citep{Sun-Yan-Zhang-Lu-2014}, and that the LAGR estimates
of true zero coefficients go to zero with probability one. Thus, selection
and estimation by LAGR has the oracle property.


\paragraph{A note on rates}

To prove the oracle properties of LAGR, we assumed that $h^{-1}n^{-1/2}a_{n}\xrightarrow{p}0$
and $hn^{-1/2}b_{n}\xrightarrow{p}\infty$. Therefore, $h^{-1}n^{-1/2}\lambda_{n}(\bm{s})\to0$
for $j\le p_{0}$ and $hn^{-1/2}\lambda_{n}(\bm{s})\|\bm{\zeta}_{j}(\bm{s})\|^{-\gamma}\to\infty$
for $j>p_{0}$.

We require that $\lambda_{n}(\bm{s})$ can satisfy both assumptions.
Suppose $\lambda_{n}(\bm{s})=n^{\alpha}$, and recall that $h=O(n^{-1/6})$
and $\|\tilde{\bm{\zeta}}_{p}(\bm{s})\|=O(h^{-1}n^{-1/2})$. Then
$h^{-1}n^{-1/2}\lambda_{n}(\bm{s})=O(n^{-1/3+\alpha})$ and $hn^{-1/2}\lambda_{n}(\bm{s})\|\tilde{\bm{\zeta}}_{p}(\bm{s}\|^{-\gamma}=O(n^{-2/3+\alpha+\gamma/3})$.

So $(2-\gamma)/3<\alpha<1/3$, which can only be satisfied for $\gamma>1$.

\begin{comment}

\section{Extension to GLLMs\label{sec:lagr-gllm}}


\subsection{Model}

Generalized linear models (GLM) extend the linear model to distributions
other than gaussian. The generalized local linear model (GLLM) is
an extension of the GLM to varying coefficient models via local regression.

As was the case for local linear regression models, the GLLM coefficients
are smooth functions of location, called $\bm{\beta}(\bm{s})$. If
the response variable $y$ is from an exponential-family distribution
then its density is 

\[
f\left\{ y(\bm{s})|\bm{x}(\bm{s})\right\} =c\left\{ y(\bm{s})\right\} \times\exp\left[\frac{\theta(\bm{s})y(\bm{s})-b\left\{ \theta(\bm{s})\right\} }{a\left\{ \phi(\bm{s})\right\} }\right]
\]


where $\phi$ and $\theta$ are parameters and

\begin{align*}
E\left\{ y(\bm{s})|\bm{x}(\bm{s})\right\} = & \mu(\bm{s})=b'\left\{ \theta(\bm{s})\right\} \\
\theta(\bm{s})= & (g\circ b')^{-1}\left\{ \eta(\bm{s})\right\} \\
\eta(\bm{s})= & \bm{x}^{T}(\bm{s})\bm{\beta}(\bm{s})=g\left\{ \mu(\bm{s})\right\} \\
\text{\text{Var}}\left\{ y(\bm{s})|\bm{x}(\bm{s})\right\} = & b''\left\{ \theta(\bm{s})\right\} a\left\{ \phi(\bm{s})\right\} 
\end{align*}


The function $g(\cdot)$ is called the link function. If its inverse
$g^{-1}(\cdot)=b'(\cdot)$ then the composition $(g\circ b')(\cdot)$
is the identity function. This particular choice of $g$ is called
the canonical link. We follow the practice of \citet{Fan-Heckman-Wand-1995}
in assuming the use of the canonical link because it is simple and
because using an alternative link function would hardly affect the
local fit.

Under the canonical link function, the expressions for the mean and
variance of the response variable can be simplified to

\begin{align*}
E\left\{ y(\bm{s})|\bm{x}(\bm{s})\right\} = & g^{-1}\left\{ \eta(\bm{s})\right\} \\
\text{\text{Var}}\left\{ y(\bm{s})|\bm{x}(\bm{s})\right\} = & a\left\{ \phi(\bm{s})\right\} /g'\left\{ \mu(\bm{s})\right\} =V\left\{ \mu(\bm{s})\right\} \times a\left\{ \phi(\bm{s})\right\} 
\end{align*}
 


\subsection{Local quasi-likelihood}

Assuming the canonical link, all that is required is to specify the
mean-variance relationship via the variance function, $V\left\{ \mu(\bm{s})\right\} $.
Then the GLLM coefficients can be estimated by maximizing the local
quasi-likelihood 

\begin{align}
\ell\left\{ \bm{\zeta}(\bm{s})\right\}  & =\sum_{i=1}^{n}K_{h}(\|\bm{s}-\bm{s}_{i}\|)Q\left[g^{-1}\left\{ z'(\bm{s}_{i})\bm{\zeta}(\bm{s})\right\} ,Y(\bm{s}_{i})\right].
\end{align}


The local quasi-likelihood generalizes the local log-likelihood that
was used to estimate coefficients in the local linear model case.
The quasi-likelihood is convex, and is defined in terms of its derivative,
the quasi-score function

\[
\frac{\partial}{\partial\mu}Q(\mu,y)=\frac{y-\mu}{V(\mu)}.
\]



\subsection{Estimation}

Under these conditions, the local quasi-likelihood is maximized where

\begin{align}
\left(\frac{\partial}{\partial\bm{\zeta}}\ell\right)\left\{ \hat{{\bm{\zeta}}}(\bm{s})\right\}  & =\sum_{i=1}^{n}K_{h}(\|\bm{s}-\bm{s}_{i}\|)\frac{y(\bm{s}_{i})-\hat{\mu}(\bm{s}_{i})}{V\left\{ \hat{\mu}(\bm{s}_{i})\right\} g'\left\{ \hat{\mu}(\bm{s}_{i})\right\} }\bm{z}(\bm{s}_{i})=\bm{0}_{3p}.
\end{align}


Except for the $K_{h}(\|\bm{s}-\bm{s}_{i}\|)$ term, this is the same
as the normal equations for estimating coefficients in a GLM. The
method of iteratively reweighted least squares (IRLS) is used to solve
for $\hat{{\bm{\zeta}}}(\bm{s})$.


\subsection{Distribution of the local coefficients}

The asymptotic distribution of the local coefficients in a varying-coefficients
GLM with a one-dimensional effect-modifying parameter are given in
\citet{Cai-Fan-Li-2000}. For coefficients that vary in two dimensions
(e.g. spatial location), the asymptotic distribution under the canonical
link is

\[
\sqrt{{nh^{2}f(\bm{{s}})}}\left[\hat{\bm{\beta}}(\bm{s})-\bm{\beta}(\bm{s})-(1/2)\kappa_{0}^{-1}\kappa_{2}h^{2}\left\{ \bm{\beta}_{uu}(\bm{s})+\bm{\beta}_{vv}(\bm{s})\right\} \right]\xrightarrow{{D}}N\left\{ \bm{0},\kappa_{0}^{-2}\nu_{0}\Gamma^{-1}(\bm{s})\right\} 
\]


where $\Gamma(\bm{s})=E\left[V\left\{ \mu(\bm{s})\right\} X(\bm{s})X(\bm{s})^{T}\right]$.


\subsection{LAGR penalty}

As in the case of linear models, the LAGR for GLMs is a grouped $\ell_{1}$
regularization method. Now, though, we use a penalized local quasi-likelihood:

\begin{align}
Q\{\bm{\zeta}(\bm{s})\} & =\mathcal{\ell}\left\{ \bm{\zeta}(\bm{s})\right\} +\mathcal{J}\{\bm{\zeta}(\bm{s})\}\notag\label{eq:adaptive-lasso-GLLM}\\
 & =\sum_{i=1}^{n}K_{h}(\|\bm{s}-\bm{s}_{i}\|)Q\left[g^{-1}\left\{ z'(\bm{s}_{i})\bm{\zeta}(\bm{s})\right\} ,Y(\bm{s}_{i})\right]+\sum_{j=1}^{p}\phi_{j}(\bm{s})\|\bm{\zeta}_{j}(\bm{s})\|
\end{align}


and similarly to the case for gaussian data, $\phi_{j}(\bm{s})=\lambda_{n}(\bm{s})\|\tilde{\bm{\zeta}}_{j}(\bm{s})\|^{-\gamma}$,
where $\lambda_{n}(\bm{s})>0$ is a the local tuning parameter applied
to all coefficients at location $\bm{s}$ and $\tilde{\bm{\zeta}}_{j}(\bm{s})$
is the vector of unpenalized local coefficients.


\subsection{Oracle properties of LAGR in the GLM setting}

The oracle properties for LAGR in the GLM setting are similar to those
in the gaussian setting:
\begin{thm}[Asymptotic normality]
\label{theorem:normality-glm} 



If $h^{-1}n^{-1/2}a_{n}\xrightarrow{p}0$ and $hn^{-1/2}b_{n}\xrightarrow{p}\infty$
then 
\[
h\sqrt{n}\left[\hat{\bm{\beta}}_{(a)}(\bm{s})-\bm{\beta}_{(a)}(\bm{s})-\frac{\kappa_{2}h^{2}}{2\kappa_{0}}\{\nabla_{uu}^{2}\bm{\beta}_{(a)}(\bm{s})+\nabla_{vv}^{2}\bm{\beta}_{(a)}(\bm{s})\}\right]\xrightarrow{d}N(0,f(\bm{s})^{-1}\kappa_{0}^{-2}\nu_{0}\Gamma^{-1}(\bm{s}))
\]

\end{thm}

\begin{thm}[Selection consistency]
\label{theorem:selection-glm}



If $h^{-1}n^{-1/2}a_{n}\xrightarrow{p}\infty$ and $hn^{-1/2}b_{n}\xrightarrow{p}\infty$
then $P\left\{ \|\hat{\bm{\zeta}}_{j}(\bm{s})\|=0\right\} \to0$ if
$j\le p_{0}$ and $P\left\{ \|\hat{\bm{\zeta}}_{j}(\bm{s})\|=0\right\} \to1$
if $j>p_{0}$. \end{thm}
\end{comment}



\section{Simulations\label{sec:simulations}}

A simulation study was conducted to assess the performance of the
method described in Sections \ref{sec:vcr}--\ref{sec:lagr-gaussian}. 

Data were simulated on the domain $[0,1]^{2}$, which was divided
into a $30\times30$ grid. Each of $p=5$ covariates $X_{1},\dots,X_{5}$
was simulated by a Gaussian random field with mean zero and exponential
covariance function $\text{Cov}\left(X_{ji},X_{ji'}\right)=\sigma_{x}^{2}\exp{\left(-\tau_{x}^{-1}\delta_{ii'}\right)}$
where $\sigma_{x}^{2}=1$ is the variance, $\tau_{x}=0.1$ is the
range parameter, and $\delta_{ii'}$ is the Euclidean distance $\|\bm{s}_{i}-\bm{s}_{i'}\|_{2}$. 

Correlation was induced between the covariates by multiplying the
matrix $\bm{X}=\left(X_{1}\cdots X_{5}\right)$ by $\bm{R}$, where
$\bm{R}$ is the Cholesky decomposition of the covariance matrix $\bm{\Sigma}=\bm{R}'\bm{R}$.
The covariance matrix $\bm{\Sigma}$ is a $5\times5$ matrix that
has ones on the diagonal and $\rho$ for all off-diagonal entries,
where $\rho$ is the between-covariate correlation. 

The simulated response was $y_{i}=\bm{x}'_{i}\bm{\beta}_{i}+\varepsilon_{i}$
for $i=1,\dots,n$ where $n=900$ and the $\varepsilon_{i}$'s were
iid Gaussian with mean zero and variance $\sigma_{\varepsilon}^{2}$.
The simulated data included the response $y$ and five covariates
$x_{1},\dots,x_{5}$. The true data-generating model uses only $x_{1}$.
The variables $x_{2},\dots,x_{5}$ are included to assess performance
in model selection. 

Three different functions were used for the coefficient surface $\beta_{1}(\bm{s})$.
They are plotted in Figure \ref{fig:simulation-coefficient-functions},
and their mathematical forms are listed in (\ref{eq:simulation-coefficient-functions}).
The first is a step function, which is equal to zero in 40\% of the
spatial domain, equal to one in a different 40\% of the spatial domain,
and increases linearly in the middle 20\% of the domain. The second
is a gradient function, which increases linearly from zero at one
end of the domain to one at the other. The final coefficient function
is a parabola taking its maximum value of 1 at the center of the domain
and falling to zero at each corner of the domain. 

\begin{figure}
\includegraphics[width=0.33\textwidth]{0_Users_wesley_git_gwr_figures_simulation_step.pdf}\includegraphics[width=0.33\textwidth]{1_Users_wesley_git_gwr_figures_simulation_gradient.pdf}\includegraphics[width=0.33\textwidth]{2_Users_wesley_git_gwr_figures_simulation_parabola.pdf}

\protect\caption{These are, respectively, the step, gradient, and parabola functions
that were used for the coefficient function $\beta_{1}(\bm{s})$ in
the VCR model $y(\bm{s}_{i})=x_{1}(\bm{s}_{i})\beta_{1}(\bm{s}_{i})+\varepsilon(\bm{s}_{i})$
when generating the data for the simulation study.\label{fig:simulation-coefficient-functions}}
\end{figure}


\begin{align}
\beta_{step}(\bm{s})= & \ \ \begin{cases}
1 & if\ s_{x}>0.6\\
5s_{x}-2 & if\ 0.4<s_{x}\le0.6\\
0 & o.w.
\end{cases}\nonumber \\
\beta_{gradient}(\bm{s})= & \ \ s_{x}\nonumber \\
\beta_{parabola}(\bm{s})= & \ \ 1-\frac{(s_{x}-0.5)^{2}+(s_{y}-0.5)^{2}}{0.5}\label{eq:simulation-coefficient-functions}
\end{align}


In total, three parameters were varied to produce 18 settings, each
of which was simulated 100 times. There were three functional forms
for the coefficient surface $\beta_{1}(\bm{s})$; data was simulated
both with low ($\rho=0$), medium ($\rho=0.5$), and high ($\rho=0.9$)
correlation between the covariates; and simulations were made with
low ($\sigma_{\varepsilon}^{2}=0.25$) and high ($\sigma_{\varepsilon}^{2}=1$)
variance for the random error term. The simulation settings are enumerated
in Table \ref{tab:simulation-settings}. 

\begin{table}
\begin{tabular}{c|c|c}
$\beta_{1}(\bm{s})$ &
$\rho$ &
$\sigma_{\varepsilon}^{2}$\tabularnewline
\hline 
\multirow{6}{*}{step} &
\multirow{2}{*}{0} &
0.25\tabularnewline
\cline{3-3} 
 &  & 1\tabularnewline
\cline{2-3} 
 & \multirow{2}{*}{0.5} &
0.25\tabularnewline
\cline{3-3} 
 &  & 1\tabularnewline
\cline{2-3} 
 & \multirow{2}{*}{0.9} &
0.25\tabularnewline
\cline{3-3} 
 &  & 1\tabularnewline
\hline 
\multirow{6}{*}{gradient} &
\multirow{2}{*}{0} &
0.25\tabularnewline
\cline{3-3} 
 &  & 1\tabularnewline
\cline{2-3} 
 & \multirow{2}{*}{0.5} &
0.25\tabularnewline
\cline{3-3} 
 &  & 1\tabularnewline
\cline{2-3} 
 & \multirow{2}{*}{0.9} &
0.25\tabularnewline
\cline{3-3} 
 &  & 1\tabularnewline
\hline 
\multirow{6}{*}{parabola} &
\multirow{2}{*}{0} &
0.25\tabularnewline
\cline{3-3} 
 &  & 1\tabularnewline
\cline{2-3} 
 & \multirow{2}{*}{0.5} &
0.25\tabularnewline
\cline{3-3} 
 &  & 1\tabularnewline
\cline{2-3} 
 & \multirow{2}{*}{0.9} &
0.25\tabularnewline
\cline{3-3} 
 &  & 1\tabularnewline
\end{tabular}

\protect\caption{Listing of the simulation settings used to assess the performance
of LAGR models versus oracle selection and no selection.\label{tab:simulation-settings}}


\end{table}



\subsection{Methods for comparison}

The performance of LAGR was compared to that of a VCR model without
variable selection, and to a VCR model with oracular selection. Oracular
selection means that exactly the correct set of covariates was used
to fit each local model.


\subsection{Results}

The results are presented in terms of the mean integrated squared
error (MISE) of the coefficient surface estimates $\hat{\beta}_{1}(\bm{s}),\dots,\hat{\beta}_{5}(\bm{s})$,
the MISE of the fitted response $\hat{y}(\bm{s})$, and the frequency
with which the coefficient surface estimates $\hat{\beta}_{1}(\bm{s}),\dots,\hat{\beta}_{5}(\bm{s})$
in the LAGR model were zero.

The MISE of the estimates of $\beta_{1}(\bm{s})$ are in Table \ref{tab:x1-mise}.

% latex table generated in R 3.1.0 by xtable 1.7-3 package
% Thu Jun 26 22:10:01 2014
\begin{table}
\centering
\begin{tabular}{rrrr}
  & LAGR & none & oracle \\ 
  \hline
1 & \emph{0.02} & 0.02 & \textbf{0.01} \\ 
  2 & \emph{0.03} & 0.03 & \textbf{0.02} \\ 
  3 & \emph{0.02} & 0.02 & \textbf{0.01} \\ 
  4 & \emph{0.03} & 0.05 & \textbf{0.02} \\ 
  5 & \emph{0.03} & 0.05 & \textbf{0.01} \\ 
  6 & \emph{0.12} & 0.17 & \textbf{0.02} \\ 
  7 & 0.01 & \emph{0.01} & \textbf{0.00} \\ 
  8 & 0.03 & \emph{0.02} & \textbf{0.01} \\ 
  9 & 0.01 & \emph{0.01} & \textbf{0.00} \\ 
  10 & 0.04 & \emph{0.03} & \textbf{0.01} \\ 
  11 & \emph{0.03} & 0.04 & \textbf{0.00} \\ 
  12 & \emph{0.14} & 0.14 & \textbf{0.01} \\ 
  13 & 0.01 & \emph{0.01} & \textbf{0.01} \\ 
  14 & 0.03 & \emph{0.02} & \textbf{0.02} \\ 
  15 & 0.01 & \emph{0.01} & \textbf{0.01} \\ 
  16 & 0.03 & \emph{0.03} & \textbf{0.02} \\ 
  17 & \emph{0.02} & 0.04 & \textbf{0.01} \\ 
  18 & 0.17 & \emph{0.14} & \textbf{0.02} \\ 
  \end{tabular}
\caption{The MISE for the estimates of $\beta_1(\bm{s})$ in each simulation setting, under variable selection via LAGR, no variable selection, and oracular variable selection. Highlighting indicates the \textbf{lowest} and \emph{next-lowest} MISE.} 
\label{tab:x1-mise}
\end{table}


Recall that $\beta_{2}(\bm{s}),\dots,\beta_{5}(\bm{s})$ are exactly
zero across the entire domain. Oracle selection will estimate these
coefficients perfectly, so we focus on the comparison between estimation
by LAGR and by the VCR model with no selection. The MISE of the estimates
of $\beta_{2}(\bm{s}),\dots,\beta_{5}(\bm{s})$ for each simulation
setting are enumerated in Table \ref{tab:x2-x5-mise}, which shows
that for every simulation setting, LAGR selection and estimation is
more accurate than the standard VCR model.

% latex table generated in R 3.1.0 by xtable 1.7-3 package
% Thu Jun 26 22:10:08 2014
\begin{table}
\centering
\begin{tabular}{rrr}
  & LAGR & none \\ 
  \hline
1 & \textbf{0.000} & 0.005 \\ 
  2 & \textbf{0.001} & 0.019 \\ 
  3 & \textbf{0.000} & 0.008 \\ 
  4 & \textbf{0.002} & 0.030 \\ 
  5 & \textbf{0.003} & 0.041 \\ 
  6 & \textbf{0.017} & 0.150 \\ 
  7 & \textbf{0.000} & 0.005 \\ 
  8 & \textbf{0.001} & 0.018 \\ 
  9 & \textbf{0.000} & 0.008 \\ 
  10 & \textbf{0.002} & 0.032 \\ 
  11 & \textbf{0.004} & 0.037 \\ 
  12 & \textbf{0.018} & 0.147 \\ 
  13 & \textbf{0.000} & 0.005 \\ 
  14 & \textbf{0.001} & 0.018 \\ 
  15 & \textbf{0.000} & 0.008 \\ 
  16 & \textbf{0.002} & 0.031 \\ 
  17 & \textbf{0.004} & 0.038 \\ 
  18 & \textbf{0.027} & 0.146 \\ 
  \end{tabular}
\caption{The MISE for the estimates of $\beta_2(\bm{s}),\dots,\beta_5(\bm{s})$ in each simulation setting, under variable selection via LAGR and no variable selection. Highlighting indicates the \textbf{lowest} MISE.} 
\label{tab:x2-x5-mise}
\end{table}


% latex table generated in R 3.1.0 by xtable 1.7-3 package
% Thu Jun 26 22:10:10 2014
\begin{table}
\centering
\begin{tabular}{rr}
  & Frequency of exact zero \\ 
  \hline
1 & 0.97 \\ 
  2 & 0.96 \\ 
  3 & 0.96 \\ 
  4 & 0.92 \\ 
  5 & 0.86 \\ 
  6 & 0.85 \\ 
  7 & 0.96 \\ 
  8 & 0.95 \\ 
  9 & 0.94 \\ 
  10 & 0.92 \\ 
  11 & 0.80 \\ 
  12 & 0.85 \\ 
  13 & 0.97 \\ 
  14 & 0.94 \\ 
  15 & 0.95 \\ 
  16 & 0.88 \\ 
  17 & 0.79 \\ 
  18 & 0.78 \\ 
  \end{tabular}
\caption{Proportion of local models under each setting in which the coefficients $\beta_2(\bm{s}),\dots,\beta_5(\bm{s})$ are estimated as exactly zero.} 
\label{tab:pzero}
\end{table}


From Table \ref{tab:pzero} we see that LAGR has good ability to identify
non-important predictors. The frequency with which $\beta_{2}(\bm{s}),\dots,\beta_{5}(\bm{s})$
were dropped from the LAGR models ranged from 0.78
to 0.97.

% latex table generated in R 3.1.0 by xtable 1.7-3 package
% Thu Jun 26 22:10:12 2014
\begin{table}
\centering
\begin{tabular}{rrrr}
  & LAGR & none & oracle \\ 
  \hline
1 & \emph{0.25} & 0.26 & \textbf{0.25} \\ 
  2 & \emph{1.00} & \textbf{1.00} & 0.99 \\ 
  3 & \emph{0.26} & 0.26 & \textbf{0.25} \\ 
  4 & \emph{0.99} & \textbf{1.00} & 0.98 \\ 
  5 & \emph{0.27} & 0.30 & \textbf{0.25} \\ 
  6 & \emph{1.08} & 1.14 & \textbf{0.98} \\ 
  7 & \emph{0.25} & \textbf{0.25} & 0.25 \\ 
  8 & \textbf{0.99} & \emph{0.99} & 0.97 \\ 
  9 & \emph{0.25} & \textbf{0.25} & 0.24 \\ 
  10 & \emph{1.00} & \textbf{1.00} & 0.97 \\ 
  11 & \emph{0.27} & 0.28 & \textbf{0.24} \\ 
  12 & \emph{1.09} & 1.12 & \textbf{0.97} \\ 
  13 & \emph{0.25} & \textbf{0.25} & 0.25 \\ 
  14 & \textbf{1.00} & \emph{1.00} & 0.98 \\ 
  15 & \textbf{0.25} & 0.25 & \emph{0.25} \\ 
  16 & \textbf{1.00} & \emph{1.00} & 0.97 \\ 
  17 & \emph{0.26} & 0.28 & \textbf{0.24} \\ 
  18 & 1.13 & \emph{1.12} & \textbf{0.98} \\ 
  \end{tabular}
\caption{The MISE for the fitted output in each simulation setting, under variable selection via LAGR, no variable selection, and oracular variable selection. Highlighting indicates the \textbf{closest} and \emph{next-closest} to the actual error variance $\sigma_\varepsilon^2$ for that setting.} 
\label{tab:misey}
\end{table}


The MISE of the fitted $\hat{y}(\bm{s})$ is listed in Table \ref{tab:misey},
where the highlighting is based on which methods estimate an error
variance that is closest to the known truth for the simulation. The
results are all very similar to each other, indicating that no method
was consistently better than the others in this simulation at fitting
the model output.


\subsection{Discussion}

The proposed LAGR method was accurate in selection and estimation,
with estimation accuracy for $\beta_{1}(\bm{s})$ about equal to that
of the VCR model with no selection, and with consistently better accuracy
for estimating $\beta_{2}(\bm{s}),\dots,\beta_{5}(\bm{s})$.

There was minimal difference in the performance of the proposed LAGR
method between low ($\sigma_{\varepsilon}=0.5$) and high ($\sigma_{\varepsilon}=1$)
error variance, and between no ($\rho=0$) and moderate ($\rho=0.5$)
correlation among the predictor variables. But the selection and estimation
accuracy did decline when there was high ($\rho=0.9$) correlation
among the predictor variables.


\section{Data example\label{sec:example}}

The proposed LAGR estimation method was used to estimate the coefficients
in a VCR model of the effect of some covariates on the price of homes
in Boston. The data source is the Boston house price data set of \citet{Harrison-Rubinfeld-1978,Gilley-Pace-1996,Pace-Gilley-1997},
which is based on the 1970 U.S. census. In the data, we have the median
price of homes sold in 506 census tracts (MEDV), along with some potential
predictor variables. The predictor variables are CRIM (the per-capita
crime rate in the tract), RM (the mean number of rooms for houses
sold in the tract), RAD (an index of how accessible the tract is from
Boston's radial roads), TAX (the property tax per \$10,000 of property
value), and LSTAT (the percentage of the tract's residents who are
considered ``lower status'').

The bandwidth parameter was set to 0.2 for a nearest neighbors-type
bandwidth, meaning that the sum of kernel weights for each local model
was 20\% of the total number of observations. The kernel used was
the Epanechnikov kernel.


\subsection{Results}

Estimates of the regression coefficients are plotted in Figure \ref{fig:boston-lagr-coefs}.

\begin{figure}

\includegraphics[width=\maxwidth]{figure/boston-plots} 


\caption{The coefficients for the boston house price data as estimated by LAGR.\label{fig:boston-lagr-coefs}}
\end{figure}

One interesting result is that LAGR indicates that the TAX variable
was nowhere an important predictor of the median house price. Another
is that the coefficients of CRIM and LSTAT are everywhere negative
or zero (meaning that the increasing the crime rate or proportion
of lower-status individuals reduces the median house price where the
effect is discernable) and that of RM is positive (meaning that when
the average house in a tract has more rooms, the median house will
be more expensive), but the coefficient of RAD is positive in some
areas and negative in others. This indicates that there are parts
of Boston where improved access to radial roads increases the median
house price and parts where it decreases the median house price.

There is not an obvious spatial pattern to the local coefficients
for RAD - there are more tracts with negative coefficients than positive,
and the positive coefficients do appear to be clustered, but the tracts
with positive coefficients are also adjacent to tracts with negative
coefficients. Indeed, there is not an obvious spatial pattern to any
of the coefficient surfaces except for TAX, which is zero everywhere.

% latex table generated in R 3.1.0 by xtable 1.7-3 package
% Thu Jun 26 22:10:19 2014
\begin{table}
\centering
\begin{tabular}{rrrr}
  & Mean & SD & Prop. zero \\ 
  \hline
CRIM & -0.07 & 0.08 & 0.49 \\ 
  RM & 1.92 & 1.43 & 0.02 \\ 
  RAD & -0.08 & 0.13 & 0.37 \\ 
  TAX & 0.00 & 0.00 & 1.00 \\ 
  LSTAT & -0.72 & 0.16 & 0.01 \\ 
  \end{tabular}
\caption{The mean, standard deviation, and proportion of zeros among the local coefficients in a model for the median house price in census tracts in Boston, with coefficients selected and fitted by LAGR.} 
\label{tab:boston-coefs-lagr}
\end{table}


A summary of the local coefficients is in Table \ref{tab:boston-coefs-lagr}.
It indicates that RM is the only predictor variable with a positive
mean of the local coefficients, but also that the mean of the local
coefficients of RM is the largest coefficient - at 1.92,
it is more than twice as large in magnitude as the mean local coefficient
of LSTAT (\ensuremath{-0.72}),
which is second-largest.

The coefficient of the CRIM variable was estimated to be exactly zero
at 49\%
of the locations. The percentage for the RAD variable was 37\%.

In their example using the same data, \citet{Sun-Yan-Zhang-Lu-2014}
estimated that the coefficients of RAD annd LSTAT should be constant,
at 0.36 and -0.45, respectively. That conclusion differs from our
result, which says that the mean local coefficient of RAD is actually
negative (\ensuremath{-0.08}), while
our mean fitted local coefficient for LSTAT was more negative than
the estimate of \citet{Sun-Yan-Zhang-Lu-2014}.

\appendix

\section{Proofs of theorems\label{app:proofs} }
\begin{proof}[Proof of theorem \ref{theorem:normality}]

\end{proof}
Define $V_{4}^{(n)}(\bm{u})$ to be the 
\begin{align}
\mkern-36muV_{4}^{(n)}(\bm{u}) & =Q\left\{ \bm{\zeta}(\bm{s})+h^{-1}n^{-1/2}\bm{u}\right\} -Q\left\{ \bm{\zeta}(\bm{s})\right\} \notag\label{eq:consistency}\\
 & \mkern-36mu=(1/2)\left[\bm{Y}-\bm{Z}(\bm{s})\left\{ \bm{\zeta}(\bm{s})+h^{-1}n^{-1/2}\bm{u}\right\} \right]^{T}\bm{W}(\bm{s})\left[\bm{Y}-\bm{Z}(\bm{s})\left\{ \bm{\zeta}(\bm{s})+h^{-1}n^{-1/2}\bm{u}\right\} \right]\notag\\
 & +\sum_{j=1}^{p}\phi_{j}(\bm{s})\|\bm{\zeta}_{j}(\bm{s})+h^{-1}n^{-1/2}\bm{u}_{j}\|\notag\\
 & -(1/2)\left\{ \bm{Y}-\bm{Z}(\bm{s})\bm{\zeta}(\bm{s})\right\} ^{T}\bm{W}(\bm{s})\left\{ \bm{Y}-\bm{Z}(\bm{s})\bm{\zeta}(\bm{s})\right\} -\sum_{j=1}^{p}\phi_{j}(\bm{s})\|\bm{\zeta}_{j}(\bm{s})\|\notag\\
 & \mkern-36mu=(1/2)\bm{u}^{T}\left\{ h^{-2}n^{-1}\bm{Z}^{T}(\bm{s})\bm{W}(\bm{s})\bm{Z}(\bm{s})\right\} \bm{u}-\bm{u}^{T}\left[h^{-1}n^{-1/2}\bm{Z}^{T}(\bm{s})\bm{W}(\bm{s})\left\{ \bm{Y}-\bm{Z}(\bm{s})\bm{\zeta}(\bm{s})\right\} \right]\notag\\
 & +\sum_{j=1}^{p}n^{-1/2}\phi_{j}(\bm{s})n^{1/2}\left\{ \|\bm{\zeta}_{j}(\bm{s})+h^{-1}n^{-1/2}\bm{u}_{j}\|-\|\bm{\zeta}_{j}(\bm{s})\|\right\} 
\end{align}


Note the different limiting behavior of the third term between the
cases $j\le p_{0}$ and $j>p_{0}$:


\paragraph{Case $j\le p_{0}$}

If $j\le p_{0}$ then $n^{-1/2}\phi_{j}(\bm{s})\to n^{-1/2}\lambda_{n}(\bm{s})\|\bm{\zeta}_{j}(\bm{s})\|^{-\gamma}$
and $|\sqrt{n}\left\{ \|\bm{\zeta}_{j}(\bm{s})+h^{-1}n^{-1/2}\bm{u}_{j}\|-\|\bm{\zeta}_{j}(\bm{s})\|\right\} |\le h^{-1}\|\bm{u}_{j}\|$
so 
\[
\lim\limits _{n\to\infty}\phi_{j}(\bm{s})\left(\|\bm{\zeta}_{j}(\bm{s})+h^{-1}n^{-1/2}\bm{u}_{j}\|-\|\bm{\zeta}_{j}(\bm{s})\|\right)\le h^{-1}n^{-1/2}\phi_{j}(\bm{s})\|\bm{u}_{j}\|\le h^{-1}n^{-1/2}a_{n}\|\bm{u}_{j}\|\to0
\]



\paragraph{Case $j>p_{0}$}

If $j>p_{0}$ then $\phi_{j}(\bm{s})\left(\|\bm{\zeta}_{j}(\bm{s})+h^{-1}n^{-1/2}\bm{u}_{j}\|-\|\bm{\zeta}_{j}(\bm{s})\|\right)=\phi_{j}(\bm{s})h^{-1}n^{-1/2}\|\bm{u}_{j}\|$.

And note that $h=O(n^{-1/6})$ so that if $hn^{-1/2}b_{n}\xrightarrow{p}\infty$
then $h^{-1}n^{-1/2}b_{n}\xrightarrow{p}\infty$.

Now, if $\|\bm{u}_{j}\|\ne0$ then 
\[
h^{-1}n^{-1/2}\phi_{j}(\bm{s})\|\bm{u}_{j}\|\ge h^{-1}n^{-1/2}b_{n}\|\bm{u}_{j}\|\to\infty
\]
. On the other hand, if $\|\bm{u}_{j}\|=0$ then $h^{-1}n^{-1/2}\phi_{j}(\bm{s})\|\bm{u}_{j}\|=0$.

Thus, the limit of $V_{4}^{(n)}(\bm{u})$ is the same as the limit
of $V_{4}^{*(n)}(\bm{u})$ where

\[
\mkern-72muV_{4}^{*(n)}(\bm{u})=\begin{cases}
(1/2)\bm{u}^{T}\left\{ h^{-2}n^{-1}\bm{Z}^{T}(\bm{s})\bm{W}(\bm{s})\bm{Z}(\bm{s})\right\} \bm{u}-\bm{u}^{T}\left[h^{-1}n^{-1/2}\bm{Z}^{T}(\bm{s})\bm{W}(\bm{s})\left\{ \bm{Y}-\bm{Z}(\bm{s})\bm{\zeta}(\bm{s})\right\} \right] & \mbox{ if }\|\bm{u}_{j}\|=0\;\forall j>p_{0}\\
\infty & \mbox{ otherwise }
\end{cases}.
\]


From which it is clear that $V_{4}^{*(n)}(\bm{u})$ is convex and
its unique minimizer is $\hat{\bm{u}}^{(n)}$:

\begin{align}
0 & =\left\{ h^{-2}n^{-1}\bm{Z}^{T}(\bm{s})\bm{W}(\bm{s})\bm{Z}(\bm{s})\right\} \hat{\bm{u}}^{(n)}-\left[h^{-1}n^{-1/2}\bm{Z}^{T}(\bm{s})\bm{W}(\bm{s})\left\{ \bm{Y}-\bm{Z}(\bm{s})\bm{\zeta}(\bm{s})\right\} \right]\notag\label{eq:limit}\\
\therefore\hat{\bm{u}}^{(n)} & =\left\{ n^{-1}\bm{Z}^{T}(\bm{s})\bm{W}(\bm{s})\bm{Z}(\bm{s})\right\} ^{-1}\left[hn^{-1/2}\bm{Z}^{T}(\bm{s})\bm{W}(\bm{s})\left\{ \bm{Y}-\bm{Z}(\bm{s})\bm{\zeta}(\bm{s})\right\} \right]\notag\\
\end{align}


By the epiconvergence results of \citet{Geyer-1994} and \citet{Knight-Fu-2000},
the minimizer of the limiting function is the limit of the minimizers
$\hat{\bm{u}}^{(n)}$. And since, by Lemma 2 of \citet{Sun-Yan-Zhang-Lu-2014},

\begin{equation}
\hat{\bm{u}}^{(n)}\xrightarrow{d}N\left(\frac{\kappa_{2}h^{2}}{2\kappa_{0}}\{\nabla_{uu}^{2}\bm{\zeta}_{j}(\bm{s})+\nabla_{vv}^{2}\bm{\zeta}_{j}(\bm{s})\},f(\bm{s})\kappa_{0}^{-2}\nu_{0}\sigma^{2}\Psi^{-1}\right)
\end{equation}
the result is proven.
\begin{proof}[Proof of theorem \ref{theorem:selection}]


We showed in Theorem \ref{theorem:normality} that $\hat{\bm{\zeta}}_{j}(\bm{s})\xrightarrow{p}\bm{\zeta}_{j}(\bm{s})+\frac{\kappa_{2}h^{2}}{2\kappa_{0}}\{\nabla_{uu}^{2}\bm{\zeta}_{j}(\bm{s})+\nabla_{vv}^{2}\bm{\zeta}_{j}(\bm{s})\}$,
so to complete the proof of selection consistency, it only remains
to show that $P\left\{ \hat{\bm{\zeta}}_{j}(\bm{s})=0\right\} \to1$
if $j>p_{0}$.
\end{proof}
The proof is by contradiction. Without loss of generality we consider
only the case $j=p$.

Assume $\|\hat{\bm{\zeta}}_{p}(\bm{s})\|\ne0$. Then $Q\left\{ \bm{\zeta}(\bm{s})\right\} $
is differentiable w.r.t. $\bm{\zeta}_{p}(\bm{s})$ and is minimized
where 
\begin{align}
0 & =\bm{Z}_{p}^{T}(\bm{s})\bm{W}(\bm{s})\left\{ \bm{Y}-\bm{Z}_{-p}(\bm{s})\hat{\bm{\zeta}}_{-p}(\bm{s})-\bm{Z}_{p}(\bm{s})\hat{\bm{\zeta}}_{p}(\bm{s})\right\} -\phi_{p}(\bm{s})\frac{\hat{\bm{\zeta}}_{p}(\bm{s})}{\|\hat{\bm{\zeta}}_{p}(\bm{s})\|}\notag\\
 & =\bm{Z}_{p}^{T}(\bm{s})\bm{W}(\bm{s})\left[\bm{Y}-\bm{Z}(\bm{s})\bm{\zeta}(\bm{s})-\frac{h^{2}\kappa_{2}}{2\kappa_{0}}\left\{ \nabla_{uu}^{2}\bm{\zeta}(\bm{s})+\nabla_{vv}^{2}\bm{\zeta}(\bm{s})\right\} \right]\notag\\
 & \mkern+72mu+\bm{Z}_{p}^{T}(\bm{s})\bm{W}(\bm{s})\bm{Z}_{-p}(\bm{s})\left[\bm{\zeta}_{-p}(\bm{s})+\frac{h^{2}\kappa_{2}}{2\kappa_{0}}\left\{ \nabla_{uu}^{2}\bm{\zeta}_{-p}(\bm{s})+\nabla_{vv}^{2}\bm{\zeta}_{-p}(\bm{s})\right\} -\hat{\bm{\zeta}}_{-p}(\bm{s})\right]\notag\\
 & \mkern+72mu+\bm{Z}_{p}^{T}(\bm{s})\bm{W}(\bm{s})\bm{Z}_{p}(\bm{s})\left[\bm{\zeta}_{p}(\bm{s})+\frac{h^{2}\kappa_{2}}{2\kappa_{0}}\left\{ \nabla_{uu}^{2}\bm{\zeta}_{p}(\bm{s})+\nabla_{vv}^{2}\bm{\zeta}_{p}(\bm{s})\right\} -\hat{\bm{\zeta}}_{p}(\bm{s})\right]\notag\\
 & \mkern+72mu-\phi_{p}(\bm{s})\frac{\hat{\bm{\zeta}}_{p}(\bm{s})}{\|\hat{\bm{\zeta}}_{p}(\bm{s})\|}\notag\\
\end{align}


So 
\begin{align}
\frac{h}{\sqrt{n}}\phi_{p}(\bm{s})\frac{\hat{\bm{\zeta}}_{p}(\bm{s})}{\|\hat{\bm{\zeta}}_{p}(\bm{s})\|} & =\bm{Z}_{p}^{T}(\bm{s})\bm{W}(\bm{s})\frac{h}{\sqrt{n}}\left[\bm{Y}-\bm{Z}(\bm{s})\bm{\zeta}(\bm{s})-\frac{h^{2}\kappa_{2}}{2\kappa_{0}}\left\{ \nabla_{uu}^{2}\bm{\zeta}(\bm{s})+\nabla_{vv}^{2}\bm{\zeta}(\bm{s})\right\} \right]\notag\label{eq:selection}\\
 & +\left\{ n^{-1}\bm{Z}_{p}^{T}(\bm{s})\bm{W}(\bm{s})\bm{Z}_{-p}(\bm{s})\right\} h\sqrt{n}\left[\bm{\zeta}_{-p}(\bm{s})+\frac{h^{2}\kappa_{2}}{2\kappa_{0}}\left\{ \nabla_{uu}^{2}\bm{\zeta}_{-p}(\bm{s})+\nabla_{vv}^{2}\bm{\zeta}_{-p}(\bm{s})\right\} -\hat{\bm{\zeta}}_{-p}(\bm{s})\right]\notag\\
 & +\left\{ n^{-1}\bm{Z}_{p}^{T}(\bm{s})\bm{W}(\bm{s})\bm{Z}_{p}(\bm{s})\right\} h\sqrt{n}\left[\bm{\zeta}_{p}(\bm{s})+\frac{h^{2}\kappa_{2}}{2\kappa_{0}}\left\{ \nabla_{uu}^{2}\bm{\zeta}_{p}(\bm{s})+\nabla_{vv}^{2}\bm{\zeta}_{p}(\bm{s})\right\} -\hat{\bm{\zeta}}_{p}(\bm{s})\right]
\end{align}


From Lemma 2 of \citet{Sun-Yan-Zhang-Lu-2014}, $\left\{ n^{-1}\bm{Z}_{p}^{T}(\bm{s})\bm{W}(\bm{s})\bm{Z}_{-p}(\bm{s})\right\} =O_{p}(1)$
and $\left\{ n^{-1}\bm{Z}_{p}^{T}(\bm{s})\bm{W}(\bm{s})\bm{Z}_{p}(\bm{s})\right\} =O_{p}(1)$.

From Theorem 3 of \citet{Sun-Yan-Zhang-Lu-2014}, we have that $h\sqrt{n}\left[\hat{\bm{\zeta}}_{-p}(\bm{s})-\bm{\zeta}_{-p}(\bm{s})-\frac{h^{2}\kappa_{2}}{2\kappa_{0}}\left\{ \nabla_{uu}^{2}\zeta_{-p}(\bm{s})+\nabla_{vv}^{2}\zeta_{-p}(\bm{s})\right\} \right]=O_{p}(1)$
and $h\sqrt{n}\left[\hat{\bm{\zeta}}_{p}(\bm{s})-\bm{\zeta}_{p}(\bm{s})-\frac{h^{2}\kappa_{2}}{2\kappa_{0}}\left\{ \nabla_{uu}^{2}\zeta_{p}(\bm{s})+\nabla_{vv}^{2}\zeta_{p}(\bm{s})\right\} \right]=O_{p}(1)$.

So the second and third terms of the sum in (\ref{eq:selection})
are $O_{p}(1)$.

We showed in the proof of \ref{theorem:normality} that $h\sqrt{n}\bm{Z}_{p}^{T}(\bm{s})\bm{W}(\bm{s})\left[\bm{Y}-\bm{Z}(\bm{s})\bm{\zeta}(\bm{s})-\frac{h^{2}\kappa_{2}}{2\kappa_{0}}\left\{ \nabla_{uu}^{2}\bm{\zeta}(\bm{s})+\nabla_{vv}^{2}\bm{\zeta}(\bm{s})\right\} \right]=O_{p}(1)$.

The three terms of the sum to the right of the equals sign in (\ref{eq:selection})
are $O_{p}(1)$, so for $\hat{\bm{\zeta}}_{p}(\bm{s})$ to be a solution,
we must have that $hn^{-1/2}\phi_{p}(\bm{s})\hat{\bm{\zeta}}_{p}(\bm{s})/\|\hat{\bm{\zeta}}_{p}(\bm{s})\|=O_{p}(1)$.

But since by assumption $\hat{\bm{\zeta}}_{p}(\bm{s})\ne0$, there
must be some $k\in\{1,\dots,3\}$ such that $|\hat{\zeta}_{p_{k}}(\bm{s})|=\max\{|\hat{\zeta}_{p_{k'}}(\bm{s})|:1\le k'\le3\}$.
And for this $k$, we have that $|\hat{\zeta}_{p_{k}}(\bm{s})|/\|\hat{\bm{\zeta}}_{p}(\bm{s})\|\ge1/\sqrt{3}>0$.

Now since $hn^{-1/2}b_{n}\to\infty$, we have that $hn^{-1/2}\phi_{p}(\bm{s})\hat{\bm{\zeta}}_{p}(\bm{s})/\|\hat{\bm{\zeta}}_{p}(\bm{s})\|\ge hb_{n}/\sqrt{3n}\to\infty$
and therefore the term to the left of the equals sign dominates the
sum to the right of the equals sign in (\ref{eq:selection}). So for
large enough $n$, $\hat{\bm{\zeta}}_{p}(\bm{s})\ne0$ cannot maximize
$Q$.

So $P\left\{ \hat{\bm{\zeta}}_{(b)}(\bm{s})=0\right\} \to1$. 

\bibliographystyle{chicago}
\bibliography{../../references/gwr}

\end{document}
