\documentclass[authoryear,review, 12pt]{elsarticle}



\newcommand{\maxwidth}{\textwidth}

\usepackage{alltt}
\usepackage[T1]{fontenc}
\usepackage[latin9]{inputenc}
\usepackage{geometry}
\geometry{verbose}
\setlength{\parskip}{\bigskipamount}
\setlength{\parindent}{0pt}
\usepackage{bm}
\usepackage{amsthm}
\usepackage{amsmath}
\usepackage{amssymb}
\usepackage{undertilde}
\usepackage{graphicx}
\usepackage{setspace}
\usepackage{esint}
\usepackage{booktabs}
\usepackage{color}
\usepackage{multirow}
\usepackage{natbib}

\mathchardef\mhyphen="2D % Define a "math hyphen"

\newcommand{\hlc}[2][yellow]{ {\sethlcolor{#1} \hl{#2}} }
\newcommand{\highlight}[1]{\colorbox{yellow}{$\displaystyle #1$}}

\newtheorem{thm}{Theorem}
\newtheorem{lem}{Lemma}

\newcommand\pr{\mathbf P}
\newcommand{\E}{\mathbb{E}}

\setlength{\textwidth}{6.5in}
\setlength{\textheight}{8.5in}
\textwidth=6.5in
\textheight=8.5in
\setlength{\topmargin}{0in}
\setlength{\oddsidemargin}{0in}
\setlength{\evensidemargin}{0in}

\journal{Statistica Sinica}

\begin{document}

\begin{frontmatter}

\title{Local Adaptive Grouped Regularization and its Oracle Properties}


\author[wrbrooks]{Wesley Brooks}
\ead{wrbrooks@uwalumni.com}

\author[jzhu]{Jun Zhu}
\ead{jzhu@stat.wisc.edu}

\author[zlu]{Zudi Lu}
\ead{Z.Lu@soton.ac.uk}

\address[wrbrooks]{Department of Statistics, University of Wisconsin, Madison, WI 53706}
\address[jzhu]{Department of Statistics and Department of Entomology, University of Wisconsin, Madison, WI 53706}
\address[zlu]{School of Mathematical Sciences, The University of Southampton Highfield, Southampton UK}

\begin{abstract}
Varying coefficient regression is a flexible technique for modeling data where the coefficients are functions of some effect-modifying parameter, often time or location. While there are a number of methods for variable selection in a varying coefficient regression model, the existing methods all do global selection, which includes or excludes each covariate over the entire domain of the effect-modifying parameter. Presented here is local adaptive grouped regularization, a method of local variable selection in varying coefficient regression. This method selects the variables that are associated with the response at a specific point in the domain of the effect-modifying parameter, and simultaneously estimates the coefficients of those covariates. In particular, the method applies the adaptive group Lasso in a local regression model with locally linear coefficient estimates. Oracle properties of the proposed method are established under local linear regression and local generalized linear regression. The method's finite sample properties are assessed in a simulation study. For illustration, the method is used to identify which covariates are associated with house prices in each census tract of the Boton house price data set.
\end{abstract}

\begin{keyword}
Nonparametric, variable selection
\end{keyword}

\end{frontmatter}

[body]

\bibliographystyle{chicago}
\bibliography{../../references/gwr}

\clearpage
\pagenumbering{arabic}

[appendix]

\end{document}

