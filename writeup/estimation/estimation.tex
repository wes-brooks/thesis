\batchmode
\makeatletter
\def\input@path{{/Users/wesley/git/gwr/writeup/estimation//}}
\makeatother
\documentclass[english]{article}\usepackage[]{graphicx}\usepackage[]{color}
%% maxwidth is the original width if it is less than linewidth
%% otherwise use linewidth (to make sure the graphics do not exceed the margin)
\makeatletter
\def\maxwidth{ %
  \ifdim\Gin@nat@width>\linewidth
    \linewidth
  \else
    \Gin@nat@width
  \fi
}
\makeatother

\definecolor{fgcolor}{rgb}{0.345, 0.345, 0.345}
\newcommand{\hlnum}[1]{\textcolor[rgb]{0.686,0.059,0.569}{#1}}%
\newcommand{\hlstr}[1]{\textcolor[rgb]{0.192,0.494,0.8}{#1}}%
\newcommand{\hlcom}[1]{\textcolor[rgb]{0.678,0.584,0.686}{\textit{#1}}}%
\newcommand{\hlopt}[1]{\textcolor[rgb]{0,0,0}{#1}}%
\newcommand{\hlstd}[1]{\textcolor[rgb]{0.345,0.345,0.345}{#1}}%
\newcommand{\hlkwa}[1]{\textcolor[rgb]{0.161,0.373,0.58}{\textbf{#1}}}%
\newcommand{\hlkwb}[1]{\textcolor[rgb]{0.69,0.353,0.396}{#1}}%
\newcommand{\hlkwc}[1]{\textcolor[rgb]{0.333,0.667,0.333}{#1}}%
\newcommand{\hlkwd}[1]{\textcolor[rgb]{0.737,0.353,0.396}{\textbf{#1}}}%

\usepackage{framed}
\makeatletter
\newenvironment{kframe}{%
 \def\at@end@of@kframe{}%
 \ifinner\ifhmode%
  \def\at@end@of@kframe{\end{minipage}}%
  \begin{minipage}{\columnwidth}%
 \fi\fi%
 \def\FrameCommand##1{\hskip\@totalleftmargin \hskip-\fboxsep
 \colorbox{shadecolor}{##1}\hskip-\fboxsep
     % There is no \\@totalrightmargin, so:
     \hskip-\linewidth \hskip-\@totalleftmargin \hskip\columnwidth}%
 \MakeFramed {\advance\hsize-\width
   \@totalleftmargin\z@ \linewidth\hsize
   \@setminipage}}%
 {\par\unskip\endMakeFramed%
 \at@end@of@kframe}
\makeatother

\definecolor{shadecolor}{rgb}{.97, .97, .97}
\definecolor{messagecolor}{rgb}{0, 0, 0}
\definecolor{warningcolor}{rgb}{1, 0, 1}
\definecolor{errorcolor}{rgb}{1, 0, 0}
\newenvironment{knitrout}{}{} % an empty environment to be redefined in TeX

\usepackage{alltt}
\usepackage[T1]{fontenc}
\usepackage[latin9]{inputenc}
\setlength{\parskip}{\bigskipamount}
\setlength{\parindent}{0pt}
\usepackage{verbatim}
\usepackage{bm}
\usepackage{amsthm}
\usepackage{amsmath}
\usepackage{amssymb}
\usepackage{graphicx}
\usepackage{setspace}
\doublespacing

\makeatletter
%%%%%%%%%%%%%%%%%%%%%%%%%%%%%% Textclass specific LaTeX commands.
\usepackage[natbibapa]{apacite}
\theoremstyle{plain}
\newtheorem{thm}{\protect\theoremname}

%%%%%%%%%%%%%%%%%%%%%%%%%%%%%% User specified LaTeX commands.
\usepackage{multirow}

\makeatother

\usepackage{babel}
\providecommand{\theoremname}{Theorem}
\IfFileExists{upquote.sty}{\usepackage{upquote}}{}
\begin{document}

\title{Local Adaptive Grouped Regularization and its Oracle Properties}


\author{Wesley Brooks, Jun Zhu, Zudi Lu}

\maketitle

\section{Introduction}

Whereas the coefficients in traditional linear regression are scalar
constants, the coefficients in a varying coefficients regression (VCR)
model are functions - often \emph{smooth} functions - of some effect-modifying
variable \cite{Hastie:1993a,Cleveland:1991}.

Current practice for VCR models relies on global model selection to
decide which variables should be included in the model, meaning that
predictors are identified as relevant or irrelevant over the entire
domain $\mathcal{D}$. \cite{Antoniadis:2012a} describe a method
for globally selecting the relevant predictors in a VCR model where
the coefficient functions are estimated with P-splines, and \cite{Wang-2008a}
show a method for doing global variable selection in a VCR model where
the coefficient functions are estimated by basis expansion.

Local adaptive grouped regularization (LAGR) is developed here as
a method to select only the locally relevant predictors at any specific
location $\bm{s}$ in the domain $\mathcal{D}$ of a VCR model. The
method of LAGR is applies to VCR models where the coefficients are
estimated using locally linear kernel smoothing. We show that LAGR
posesses the oracle properties of asymptotically selecting exactly
the correct local predictors and estimating their local coefficients
as accurately as would be possible if the identity of the nonzero
coefficients for the local model were known in advance.

Using kernel smoothing for nonparametric regression is described in
detail in \cite{Fan-Gijbels-1996}. The extension to estimating VCR
models is made by \cite{Fan-Zhang-1999} for a VCR a univariate effect-modifying
variable, and by \cite{Sun-Yan-Zhang-Lu-2014} for two-dimensional
effect-modifying variable and autocorrelation among the obverved response.
These methods minimize the boundary effect \cite{Hastie:1993b} by
estimating the coefficients as local polynomials of odd degree (usually
locally linear).

For linear regression models, the adaptive lasso (AL) \cite{Zou-2006}
produces consistent estimates of the coefficients and has been shown
to have appealing properties for automating variable selection, which
under suitable conditions include the ``oracle'' property of asymptotically
including exactly the correct set of covariates and estimating their
coefficients as well as if the correct covariates were known in advance.
For data where the obvserved variables fall into mutually exclusive
groups that are known in advance, the adaptive group lasso \cite{Yuan-Lin-2006,Wang-Leng-2008}
has similar oracle properties to the adaptive lasso while doing selection
at the level of groups rather than individual variables. The proposed
LAGR method uses the adaptive group lasso for local variable selection
and coefficient estimation in a locally linear regression model.

The remainder of this document is organized as follows. The kernel-based
VCR model is described in Section \ref{sec:vcr}; the proposed LAGR
technique and its oracle properties are presented in Section \ref{sec:lagr-gaussian};
in Section \ref{sec:simulations}, the performance of the proposed
LAGR technique is evaluated in a simulation study, and in Section
\ref{sec:example} the proposed method is applied to the Boston house
price dataset. Proofs of the theorems appear in Appendix \ref{app:proofs}.


\section{Varying coefficients regression\label{sec:vcr}}


\subsection{Model}

Consider $n$ data points, observed at sampling locations $\bm{s}_{i}=(s_{i,1},\; s_{i,2})^{T}$
for $i=1,\dots,n$, which are distributed in a spatial domain $\mathcal{D}\subset\mathbb{R}^{2}$
according to a density $f(\bm{s})$ with $\bm{s}\in\mathcal{D}$.
For $i=1,\dots,n$, let $y(\bm{s}_{i})$ and $\bm{x}(\bm{s}_{i})$
denote, respectively, the univariate response and the $(p+1)$-variate
vector of covariates measured at location $\bm{s}_{i}$. At each location
$\bm{s}_{i}$, assume that the outcome is related to the covariates
by a linear regression where the coefficients $\bm{\beta}(\bm{s}_{i})$
may be spatially-varying and $\varepsilon(\bm{s}_{i})$ is random
error at location $\bm{s}_{i}$. That is, 
\begin{align}
y(\bm{s}_{i})=\bm{x}(\bm{s}_{i})'\bm{\beta}(\bm{s}_{i})+\varepsilon(\bm{s}_{i}).\label{eq:lm(s)}
\end{align}


Further assume that the error term $\varepsilon(\bm{s}_{i})$ is normally
distributed with zero mean and variance $\sigma^{2}$, and that $\varepsilon(\bm{s}_{i})$,
$i=1,\dots,n$ are independent. That is, 
\begin{align}
\bm{\varepsilon}\overset{iid}{\sim}\mathcal{N}\left(0,\sigma^{2}\right).\label{eq:err}
\end{align}


In the context of nonparametric regression, the boundary-effect bias
can be reduced by local polynomial modeling, usually in the form of
a locally linear model \cite{Fan-Gijbels-1996}. Here, to prepare
for the estimation of locally linear coefficients, we augment the
local design matrix with covariate-by-location interactions in two
dimensions \cite{Wang-2008b}. The augmented local design matrix at
location $\bm{s}_{i}$ is 
\begin{align}
\bm{Z}(\bm{s}_{i})=\left(\bm{X}\:\: L_{i}\bm{X}\:\: M_{i}\bm{X}\right)
\end{align}


where $\bm{X}$ is the unaugmented matrix of covariates, $\bm{L}_{i}=\text{diag}\{s_{i',1}-s_{i,1}\}$
and $\bm{M}_{i}=\text{diag}\{s_{i',2}-s_{i,2}\}$ for $i'=1,\dots,n$.

Now we have that $Y(\bm{s}_{i})=\left\{ \bm{Z}(\bm{s}_{i})\right\} _{i}^{T}\bm{\zeta}(\bm{s}_{i})+\varepsilon(\bm{s}_{i})$,
where $\left\{ \bm{Z}(\bm{s}_{i})\right\} _{i}^{T}$ is the $i$th
row of the matrix $\bm{Z}(\bm{s}_{i})$ as a row vector, and $\bm{\zeta}(\bm{s}_{i})$
is the vector of local coefficients at location $\bm{s}_{i}$, augmented
with the local gradients of the coefficient surfaces in the two spatial
dimensions, indicated by $\nabla_{u}$ and $\nabla_{v}$:

\[
\bm{\zeta}(\bm{s}_{i})=\left(\bm{\beta}(\bm{s}_{i})^{T},\;\nabla_{u}\bm{\beta}(\bm{s}_{i})^{T},\;\nabla_{v}\bm{\beta}(\bm{s}_{i})^{T}\right)^{T}
\]



\subsection{Local Likelihood and Coefficient Estimation}

The total log-likelihood of the observed data is the sum of the log-likelihood
of each individual observation: 
\begin{align}
\ell\left(\bm{\zeta}\right)=-(1/2)\sum_{i=1}^{n}\left[\log{\sigma^{2}}+\sigma^{-2}\left\{ y(\bm{s}_{i})-\bm{z}'(\bm{s}_{i})\bm{\zeta}(\bm{s}_{i})\right\} ^{2}\right].\label{eq:coefficients}
\end{align}


Since there are a total of $n\times3(p+1)+1$ parameters for $n$
observations, the model is not identifiable and it is not possible
to directly maximize the total likelihood. But when the coefficient
functions are smooth, the coefficients at location $\bm{s}$ can approximate
the coefficients within some neighborhood of $\bm{s}$, with the quality
of the approximation declining as the distance from $\bm{s}$ increases.

This intuition is formalized by the local (log-)likelihood, which
is maximized at location $\bm{s}$ to estimate the local coefficients
$\bm{\zeta}(\bm{s})$:

\begin{align}
\ell\left\{ \bm{\zeta}(\bm{s})\right\}  & =-(1/2)\sum_{i=1}^{n}K_{h}(\|\bm{s}-\bm{s}_{i}\|)\left[\log{\sigma^{2}}+\sigma^{-2}\left\{ y(\bm{s}_{i})-\bm{z}'(\bm{s}_{i})\bm{\zeta}(\bm{s})\right\} ^{2}\right]\label{eq:local-log-likelihood}
\end{align}


where $h$ is a bandwidth parameter and the $K_{h}(\|\bm{s}-\bm{s}_{i}\|)$
for $i=1,\dots,n$ are local weights from a kernel function. For instance,
the Epanechnikov kernel is defined as \cite{Samiuddin-el-Sayyad-1990}:
\begin{align}
K_{h}(\|\bm{s}_{i}-\bm{s}_{i'}\|) & =h^{-2}K\left(h^{-1}\|\bm{s}_{i}-\bm{s}_{i'}\|\right)\notag\label{eq:epanechnikov}\\
K(x) & =\begin{cases}
(3/4)(1-x^{2}) & \mbox{ if }x<1,\\
0 & \mbox{ if }x\geq1.
\end{cases}
\end{align}


Letting $\bm{W}(\bm{s})=diag\left\{ K_{h}(\|\bm{s}-\bm{s}_{i}\|)\right\} $
be a diagonal matrix of kernel weights, the local likelihood is maximized
by weighted least squares: 
\begin{align}
\mathcal{S}\left\{ \bm{\zeta}(\bm{s})\right\}  & =(1/2)\left\{ \bm{Y}-\bm{Z}(\bm{s})\bm{\zeta}(\bm{s})\right\} ^{T}\bm{W}(\bm{s})\left\{ \bm{Y}-\bm{Z}(\bm{s})\bm{\zeta}(\bm{s})\right\} ^{T}\notag\label{eq:zeta-hat}
\end{align}


Thus, we have

\[
\tilde{\bm{\zeta}}(\bm{s})=\left\{ \bm{Z}^{T}(\bm{s})\bm{W}(\bm{s})\bm{Z}(\bm{s})\right\} ^{-1}\bm{Z}^{T}(\bm{s})\bm{W}(\bm{s})\bm{Y}
\]


Now Theorem 3 of \cite{Sun-Yan-Zhang-Lu-2014} says that, for any
given $\bm{{s}}$

\[
\sqrt{{nh^{2}f(\bm{{s}})}}\left[\hat{\bm{\beta}}(\bm{s})-\bm{\beta}(\bm{s})-(1/2)\kappa_{0}^{-1}\kappa_{2}h^{2}\left\{ \bm{\beta}_{uu}(\bm{s})+\bm{\beta}_{vv}(\bm{s})\right\} \right]\xrightarrow{{D}}N\left(\bm{0},\kappa_{0}^{-2}\nu_{0}\sigma^{2}\Psi^{-1}\right)
\]



\section{Local Variable Selection with LAGR\label{sec:lagr-gaussian}}


\subsection{The LAGR-Penalized Local Likelihood}

Estimating the local coefficients by (\ref{eq:zeta-hat}) relies on
\emph{a priori} variable selection. A new method of penalized regression
to simultaneously select the locally relevant predictors and estimate
the local coefficients. For this purpose, each raw covariate is grouped
with its covariate-by-location interactions. That is, $\bm{\zeta}_{j}(\bm{s})=\left(\beta_{j}(\bm{s})\;\;\;\nabla_{u}\beta_{j}(\bm{s})\;\;\;\nabla_{v}\beta_{j}(\bm{s})\right)^{T}$
for $j=1,\dots,p$. By the mechanism of the group lasso, variables
within the same group are included in or dropped from the model together.
The intercept group is left unpenalized. The proposed LAGR penalty
is an adaptive $\ell_{1}$ penalty akin to the adaptive group lasso
\cite{Wang-Leng-2008,Zou-2006}.

More specifically, we consider the penalized local sum of squares
at location $\bm{s}$: 
\begin{align}
\mathcal{J}\{\bm{\zeta}(\bm{s})\} & =\mathcal{S}\left\{ \bm{\zeta}(\bm{s})\right\} +\mathcal{P}\left\{ \bm{\zeta}(\bm{s})\right\} \notag\label{eq:adaptive-lasso-WLS}
\end{align}


where $\mathcal{S}\left\{ \bm{\zeta}(\bm{s})\right\} =(1/2)\left\{ \bm{Y}-\bm{Z}(\bm{s})\bm{\zeta}(\bm{s})\right\} ^{T}\bm{W}(\bm{s})\left\{ \bm{Y}-\bm{Z}(\bm{s})\bm{\zeta}(\bm{s})\right\} ^{T}$
is the locally weighted sum of squares, $\mathcal{P}\left\{ \bm{\zeta}(\bm{s})\right\} =\sum_{j=1}^{p}\phi_{j}(\bm{s})\|\bm{\zeta}_{j}(\bm{s})\|$
is a local adaptive grouped regularization (LAGR) penalty, and $\|\cdot\|$
is the $L_{2}$-norm.

The LAGR penalty for the $j$th group of coefficients $\bm{\zeta}_{j}(\bm{s})$
at location $\bm{s}$ is $\phi_{j}(\bm{s})=\lambda_{n}(\bm{s})\|\tilde{\bm{\zeta}}_{j}(\bm{s})\|^{-\gamma}$,
where $\lambda_{n}(\bm{s})>0$ is a local tuning parameter applied
to all coefficients at location $\bm{s}$ and $\tilde{\bm{\zeta}}_{j}(\bm{s})$
is the vector of unpenalized local coefficients from (\ref{eq:zeta-hat}).


\subsection{Oracle properties of LAGR}
\begin{thm}[Asymptotic normality]
\label{theorem:normality} 



If $h^{-1}n^{-1/2}a_{n}\xrightarrow{p}0$ and $hn^{-1/2}b_{n}\xrightarrow{p}\infty$
then 
\[
h\sqrt{n}\left[\hat{\bm{\beta}}_{(a)}(\bm{s})-\bm{\beta}_{(a)}(\bm{s})-\frac{\kappa_{2}h^{2}}{2\kappa_{0}}\{\nabla_{uu}^{2}\bm{\beta}_{(a)}(\bm{s})+\nabla_{vv}^{2}\bm{\beta}_{(a)}(\bm{s})\}\right]\xrightarrow{d}N(0,f(\bm{s})^{-1}\kappa_{0}^{-2}\nu_{0}\sigma^{2}\Psi^{-1})
\]

\end{thm}

\begin{thm}[Selection consistency]
\label{theorem:selection}



If $h^{-1}n^{-1/2}a_{n}\xrightarrow{p}\infty$ and $hn^{-1/2}b_{n}\xrightarrow{p}\infty$
then $P\left\{ \|\hat{\bm{\zeta}}_{j}(\bm{s})\|=0\right\} \to0$ if
$j\le p_{0}$ and $P\left\{ \|\hat{\bm{\zeta}}_{j}(\bm{s})\|=0\right\} \to1$
if $j>p_{0}$. 
\end{thm}

\paragraph{Remarks}

Together, Theorem�\ref{theorem:normality} and Theorem \ref{theorem:selection}
indicate that the LAGR estimates have the same asymptotic distribution
as a local regression model where the nonzero coefficients are known
in advance \cite{Sun-Yan-Zhang-Lu-2014}, and that the LAGR estimates
of true zero coefficients go to zero with probability one. Thus, selection
and estimation by LAGR has the oracle property.


\paragraph{A note on rates}

To prove the oracle properties of LAGR, we assumed that $h^{-1}n^{-1/2}a_{n}\xrightarrow{p}0$
and $hn^{-1/2}b_{n}\xrightarrow{p}\infty$. Therefore, $h^{-1}n^{-1/2}\lambda_{n}(\bm{s})\to0$
for $j\le p_{0}$ and $hn^{-1/2}\lambda_{n}(\bm{s})\|\bm{\zeta}_{j}(\bm{s})\|^{-\gamma}\to\infty$
for $j>p_{0}$.

We require that $\lambda_{n}(\bm{s})$ can satisfy both assumptions.
Suppose $\lambda_{n}(\bm{s})=n^{\alpha}$, and recall that $h=O(n^{-1/6})$
and $\|\tilde{\bm{\zeta}}_{p}(\bm{s})\|=O(h^{-1}n^{-1/2})$. Then
$h^{-1}n^{-1/2}\lambda_{n}(\bm{s})=O(n^{-1/3+\alpha})$ and $hn^{-1/2}\lambda_{n}(\bm{s})\|\tilde{\bm{\zeta}}_{p}(\bm{s}\|^{-\gamma}=O(n^{-2/3+\alpha+\gamma/3})$.

So $(2-\gamma)/3<\alpha<1/3$, which can only be satisfied for $\gamma>1$.


\subsection{Selecting the tuning parameter $\lambda_{n}(\bm{s})$}

In practical application, it is necessary to select the LAGR tuning
parameter $\lambda_{n}(\bm{s})$ for each local model. A popular approach
in other lasso-type problems is to select the tuning parameter that
maximizes a criterion that approximates the expected log-likelihood
of a new, independent data set drawn from the same distribution. This
is the framework of Mallows' Cp \cite{Mallows-1973}, Stein's unbiased
risk estimate (SURE) \cite{Stein-1981} and Akaike's information criterion
(AIC) \cite{Akaike-1973}.

These criteria use a so-called covariance penalty \cite{Efron:2004a}
to estimate the bias due to using the same data set to select a model
and to estimate its parameters. We adopt the approximate degrees of
freedom for the adaptive group lasso from \cite{Yuan-Lin-2006} and
minimize the AICc \cite{Hurvich-1998} to select the tuning parameter
$\lambda_{n}(\bm{s})$:

\begin{align*}
\hat{df}(\lambda;\bm{s})= & \sum_{j=1}^{p}I\left(\|\hat{\bm{\zeta}}(\lambda;\bm{s})\|>0\right)+\sum_{j=1}^{p}\frac{\|\hat{\bm{\zeta}}(\lambda;\bm{s})\|}{\|\tilde{\bm{\zeta}}(\bm{s})\|}(p_{j}-1)\\
\text{AIC}_{c}(\lambda;\bm{s})= & \sum_{i=1}^{n}K_{h}(\|\bm{s}-\bm{s}_{i}\|)\sigma^{-2}\left\{ y(\bm{s}_{i})-z'(\bm{s}_{i})\hat{\bm{\zeta}}(\lambda;\bm{s})\right\} ^{2}+2\hat{df}(\lambda;\bm{s})+\frac{2\hat{df}(\lambda;\bm{s})\left\{ \hat{df}(\lambda;\bm{s})+1\right\} }{\sum_{i=1}^{n}K_{h}(\|\bm{s}-\bm{s}_{i}\|)-\hat{df}(\lambda;\bm{s})-1}
\end{align*}


where the local coefficient estimate has been written $\hat{\bm{\zeta}}(\lambda;\bm{s})$
to emphasize that it depends on the tuning parameter.

\begin{comment}

\section{Extension to GLLMs\label{sec:lagr-gllm}}


\subsection{Model}

Generalized linear models (GLM) extend the linear model to distributions
other than gaussian. The generalized local linear model (GLLM) is
an extension of the GLM to varying coefficient models via local regression.

As was the case for local linear regression models, the GLLM coefficients
are smooth functions of location, called $\bm{\beta}(\bm{s})$. If
the response variable $y$ is from an exponential-family distribution
then its density is 

\[
f\left\{ y(\bm{s})|\bm{x}(\bm{s}),\theta\left(\bm{s}\right)\right\} =c\left\{ y(\bm{s})\right\} \times\left[\exp\theta(\bm{s})y(\bm{s})-b\left\{ \theta(\bm{s})\right\} \right]
\]


where $\phi$ and $\theta$ are parameters and

\begin{align*}
E\left\{ y(\bm{s})|\bm{x}(\bm{s})\right\} = & \mu(\bm{s})=b'\left\{ \theta(\bm{s})\right\} \\
\theta(\bm{s})= & (g\circ b')^{-1}\left\{ \eta(\bm{s})\right\} \\
\eta(\bm{s})= & \bm{x}^{T}(\bm{s})\bm{\beta}(\bm{s})=g\left\{ \mu(\bm{s})\right\} \\
\text{\text{Var}}\left\{ y(\bm{s})|\bm{x}(\bm{s})\right\} = & b''\left\{ \theta(\bm{s})\right\} 
\end{align*}


The function $g(\cdot)$ is called the link function. If its inverse
$g^{-1}(\cdot)=b'(\cdot)$ then the composition $(g\circ b')(\cdot)$
is the identity function. This particular choice of $g$ is called
the canonical link. We follow the practice of \cite{Fan-Heckman-Wand-1995}
in assuming the use of the canonical link.

Under the canonical link function, the expressions for the mean and
variance of the response variable can be simplified to

\begin{align*}
E\left\{ y(\bm{s})|\bm{x}(\bm{s})\right\} = & g^{-1}\left\{ \eta(\bm{s})\right\} \\
\text{\text{Var}}\left\{ y(\bm{s})|\bm{x}(\bm{s})\right\} = & \left[g'\left\{ \mu(\bm{s})\right\} \right]^{-1}=V\left\{ \mu(\bm{s})\right\} \\
\frac{d}{d\mu} & g^{-1}\left\{ \mu\left(\bm{s}\right)\right\} =V\left\{ \mu\left(\bm{s}\right)\right\} 
\end{align*}
 


\subsection{Local quasi-likelihood}

Assuming the canonical link, all that is required is to specify the
mean-variance relationship via the variance function, $V\left\{ \mu(\bm{s})\right\} $.
Then the GLLM coefficients can be estimated by maximizing the local
quasi-likelihood 

\begin{align}
\mathcal{\ell}^{*}\left\{ \bm{\zeta}(\bm{s})\right\}  & =\sum_{i=1}^{n}K_{h}(\|\bm{s}-\bm{s}_{i}\|)Q\left[g^{-1}\left\{ \bm{z}'(\bm{s}_{i})\bm{\zeta}(\bm{s})\right\} ,Y(\bm{s}_{i})\right].
\end{align}


The local quasi-likelihood generalizes the local log-likelihood that
was used to estimate coefficients in the local linear model case.
The quasi-likelihood is convex, and is defined in terms of its derivative,
the quasi-score function

\[
\frac{\partial}{\partial\mu}Q(\mu,y)=\frac{y-\mu}{V(\mu)}.
\]



\subsection{Estimation}

Under these conditions, the local quasi-likelihood is maximized where

\begin{align}
\frac{\partial}{\partial\bm{\zeta}}\mathcal{\ell}^{*}\left\{ \hat{{\bm{\zeta}}}(\bm{s})\right\}  & =\sum_{i=1}^{n}K_{h}(\|\bm{s}-\bm{s}_{i}\|)\frac{y(\bm{s}_{i})-\hat{\mu}(\bm{s}_{i};\bm{s})}{V\left\{ \hat{\mu}(\bm{s}_{i};\bm{s})\right\} }\bm{z}(\bm{s}_{i})=\bm{0}_{3p}
\end{align}


and $\hat{\mu}\left(\bm{s}_{i};\bm{s}\right)=g^{-1}\left\{ \bm{z}'(\bm{s}_{i})\hat{\bm{\zeta}}\left(\bm{s}\right)\right\} $
is the mean at location $\bm{s}_{i}$ estimated using the coefficients
$\hat{\bm{\zeta}}\left(\bm{s}\right)$ fitted at location $\bm{s}$.
Except for the $K_{h}(\|\bm{s}-\bm{s}_{i}\|)$ term, this is the same
as the normal equations for estimating coefficients in a GLM. The
method of iteratively reweighted least squares (IRLS) is used to solve
for $\hat{{\bm{\zeta}}}(\bm{s})$.


\subsection{Distribution of the local coefficients}

The asymptotic distribution of the local coefficients in a varying-coefficients
GLM with a one-dimensional effect-modifying parameter are given in
\cite{Cai-Fan-Li-2000}. For coefficients that vary in two dimensions
(e.g. spatial location), the asymptotic distribution under the canonical
link is

\[
\sqrt{{nh^{2}f(\bm{{s}})}}\left[\hat{\bm{\beta}}(\bm{s})-\bm{\beta}(\bm{s})-(1/2)\kappa_{0}^{-1}\kappa_{2}h^{2}\left\{ \bm{\beta}_{uu}(\bm{s})+\bm{\beta}_{vv}(\bm{s})\right\} \right]\xrightarrow{{D}}N\left\{ \bm{0},\kappa_{0}^{-2}\nu_{0}\Gamma^{-1}(\bm{s})\right\} 
\]


where $\Gamma(\bm{s})=E\left[V\left\{ \mu(\bm{s})\right\} X(\bm{s})X(\bm{s})^{T}\right]$.


\subsection{LAGR penalty}

As in the case of linear models, the LAGR for GLMs is a grouped $\ell_{1}$
regularization method. Now, though, we use a penalized local quasi-likelihood:

\begin{align}
\mathcal{J}\left\{ \bm{\zeta}(\bm{s})\right\}  & =\mathcal{\ell}^{*}\left\{ \bm{\zeta}(\bm{s})\right\} +\mathcal{P}\{\bm{\zeta}(\bm{s})\}\notag\label{eq:adaptive-lasso-GLLM}\\
 & =\sum_{i=1}^{n}K_{h}(\|\bm{s}-\bm{s}_{i}\|)Q\left[g^{-1}\left\{ z'(\bm{s}_{i})\bm{\zeta}(\bm{s})\right\} ,Y(\bm{s}_{i})\right]+\sum_{j=1}^{p}\phi_{j}(\bm{s})\|\bm{\zeta}_{j}(\bm{s})\|
\end{align}


and similarly to the case for gaussian data, $\phi_{j}(\bm{s})=\lambda_{n}(\bm{s})\|\tilde{\bm{\zeta}}_{j}(\bm{s})\|^{-\gamma}$,
where $\lambda_{n}(\bm{s})>0$ is a the local tuning parameter applied
to all coefficients at location $\bm{s}$ and $\tilde{\bm{\zeta}}_{j}(\bm{s})$
is the vector of unpenalized local coefficients.


\subsection{Oracle properties of LAGR in the GLM setting}

The oracle properties for LAGR in the GLM setting are similar to those
in the gaussian setting:
\begin{thm}[Asymptotic normality]
\label{theorem:normality-glm} 



If $h^{-1}n^{-1/2}a_{n}\xrightarrow{p}0$ and $hn^{-1/2}b_{n}\xrightarrow{p}\infty$
then 
\[
h\sqrt{n}\left[\hat{\bm{\beta}}_{(a)}(\bm{s})-\bm{\beta}_{(a)}(\bm{s})-\frac{\kappa_{2}h^{2}}{2\kappa_{0}}\{\nabla_{uu}^{2}\bm{\beta}_{(a)}(\bm{s})+\nabla_{vv}^{2}\bm{\beta}_{(a)}(\bm{s})\}\right]\xrightarrow{d}N(0,f(\bm{s})^{-1}\kappa_{0}^{-2}\nu_{0}\Gamma^{-1}(\bm{s}))
\]

\end{thm}

\begin{thm}[Selection consistency]
\label{theorem:selection-glm}



If $h^{-1}n^{-1/2}a_{n}\xrightarrow{p}\infty$ and $hn^{-1/2}b_{n}\xrightarrow{p}\infty$
then $P\left\{ \|\hat{\bm{\zeta}}_{j}(\bm{s})\|=0\right\} \to0$ if
$j\le p_{0}$ and $P\left\{ \|\hat{\bm{\zeta}}_{j}(\bm{s})\|=0\right\} \to1$
if $j>p_{0}$. \end{thm}
\end{comment}



\section{Simulations\label{sec:simulations}}

A simulation study was conducted to assess the performance of the
method described in Sections \ref{sec:vcr}--\ref{sec:lagr-gaussian}. 

Data were simulated on the domain $[0,1]^{2}$, which was divided
into a $30\times30$ grid. Each of $p=5$ covariates $X_{1},\dots,X_{5}$
was simulated by a Gaussian random field with mean zero and exponential
covariance function $\text{Cov}\left(X_{ji},X_{ji'}\right)=\sigma_{x}^{2}\exp{\left(-\tau_{x}^{-1}\delta_{ii'}\right)}$
where $\sigma_{x}^{2}=1$ is the variance, $\tau_{x}=0.1$ is the
range parameter, and $\delta_{ii'}$ is the Euclidean distance $\|\bm{s}_{i}-\bm{s}_{i'}\|_{2}$. 

Correlation was induced between the covariates by multiplying the
matrix $\bm{X}=\left(X_{1}\cdots X_{5}\right)$ by $\bm{R}$, where
$\bm{R}$ is the Cholesky decomposition of the covariance matrix $\bm{\Sigma}=\bm{R}'\bm{R}$.
The covariance matrix $\bm{\Sigma}$ is a $5\times5$ matrix that
has ones on the diagonal and $\rho$ for all off-diagonal entries,
where $\rho$ is the between-covariate correlation. 

The simulated response was $y_{i}=\bm{x}'_{i}\bm{\beta}_{i}+\varepsilon_{i}$
for $i=1,\dots,n$ where $n=900$ and the $\varepsilon_{i}$'s were
iid Gaussian with mean zero and variance $\sigma_{\varepsilon}^{2}$.
The simulated data included the response $y$ and five covariates
$x_{1},\dots,x_{5}$. The true data-generating model uses only $x_{1}$.
The variables $x_{2},\dots,x_{5}$ are included to assess performance
in model selection. 

Three different functions were used for the coefficient surface $\beta_{1}(\bm{s})$.
They are plotted in Figure \ref{fig:simulation-coefficient-functions},
and their mathematical forms are listed in (\ref{eq:simulation-coefficient-functions}).
The first is a step function, which is equal to zero in 40\% of the
spatial domain, equal to one in a different 40\% of the spatial domain,
and increases linearly in the middle 20\% of the domain. The second
is a gradient function, which increases linearly from zero at one
end of the domain to one at the other. The final coefficient function
is a parabola taking its maximum value of 1 at the center of the domain
and falling to zero at each corner of the domain. 

\begin{figure}
\includegraphics[width=0.33\textwidth]{0_Users_wesley_git_gwr_figures_simulation_step.pdf}\includegraphics[width=0.33\textwidth]{1_Users_wesley_git_gwr_figures_simulation_gradient.pdf}\includegraphics[width=0.33\textwidth]{2_Users_wesley_git_gwr_figures_simulation_parabola.pdf}

\protect\caption{These are, respectively, the step, gradient, and parabola functions
that were used for the coefficient function $\beta_{1}(\bm{s})$ in
the VCR model $y(\bm{s}_{i})=x_{1}(\bm{s}_{i})\beta_{1}(\bm{s}_{i})+\varepsilon(\bm{s}_{i})$
when generating the data for the simulation study.\label{fig:simulation-coefficient-functions}}
\end{figure}


\begin{align}
\beta_{step}(\bm{s})= & \ \ \begin{cases}
1 & if\ s_{x}>0.6\\
5s_{x}-2 & if\ 0.4<s_{x}\le0.6\\
0 & o.w.
\end{cases}\nonumber \\
\beta_{gradient}(\bm{s})= & \ \ s_{x}\nonumber \\
\beta_{parabola}(\bm{s})= & \ \ 1-\frac{(s_{x}-0.5)^{2}+(s_{y}-0.5)^{2}}{0.5}\label{eq:simulation-coefficient-functions}
\end{align}


In total, three parameters were varied to produce 18 settings, each
of which was simulated 100 times. There were three functional forms
for the coefficient surface $\beta_{1}(\bm{s})$; data was simulated
both with low ($\rho=0$), medium ($\rho=0.5$), and high ($\rho=0.9$)
correlation between the covariates; and simulations were made with
low ($\sigma_{\varepsilon}^{2}=0.25$) and high ($\sigma_{\varepsilon}^{2}=1$)
variance for the random error term. The simulation settings are enumerated
in Table \ref{tab:simulation-settings}. 




\subsection{Methods for comparison}

The performance of LAGR was compared to that of a VCR model without
variable selection, and to a VCR model with oracular selection. Oracular
selection means that exactly the correct set of covariates was used
to fit each local model.


\subsection{Results}

The results are presented in terms of the mean integrated squared
error (MISE) of the coefficient surface estimates $\hat{\beta}_{1}(\bm{s}),\dots,\hat{\beta}_{5}(\bm{s})$,
the MISE of the fitted response $\hat{y}(\bm{s})$, and the frequency
with which the coefficient surface estimates $\hat{\beta}_{1}(\bm{s}),\dots,\hat{\beta}_{5}(\bm{s})$
in the LAGR model were zero.

The MISE of the estimates of $\beta_{1}(\bm{s})$ are in Table \ref{tab:x1-mise}.

\begin{table}
	\begin{tabular}{cccccc}
		$\beta_{1}(\bm{s})$ & $\rho$ & $\sigma_{\varepsilon}^{2}$ & LAGR & VCR & Oracle \\
		\hline 

\begin{kframe}


{\ttfamily\noindent\color{warningcolor}{\#\# Warning: cannot open compressed file '/Users/wesley/scratch/gwr-sim-output.RData', probable reason 'No such file or directory'}}

{\ttfamily\noindent\bfseries\color{errorcolor}{\#\# Error: cannot open the connection}}

{\ttfamily\noindent\bfseries\color{errorcolor}{\#\# Error: object 'output' not found}}

{\ttfamily\noindent\bfseries\color{errorcolor}{\#\# Error: subscript out of bounds}}

{\ttfamily\noindent\bfseries\color{errorcolor}{\#\# Error: subscript out of bounds}}

{\ttfamily\noindent\bfseries\color{errorcolor}{\#\# Error: object 'output' not found}}

{\ttfamily\noindent\bfseries\color{errorcolor}{\#\# Error: subscript out of bounds}}

{\ttfamily\noindent\bfseries\color{errorcolor}{\#\# Error: length of 'dimnames' [2] not equal to array extent}}

{\ttfamily\noindent\bfseries\color{errorcolor}{\#\# Error: number of rows of matrices must match (see arg 2)}}\end{kframe}

	\end{tabular}
	\caption{Listing of the simulation settings used to assess the performance of LAGR models versus oracle selection and no selection.}
	\label{tab:simulation-settings}
\end{table}

Recall that $\beta_{2}(\bm{s}),\dots,\beta_{5}(\bm{s})$ are exactly
zero across the entire domain. Oracle selection will estimate these
coefficients perfectly, so we focus on the comparison between estimation
by LAGR and by the VCR model with no selection. The MISE of the estimates
of $\beta_{2}(\bm{s}),\dots,\beta_{5}(\bm{s})$ for each simulation
setting are enumerated in Table \ref{tab:x2-x5-mise}, which shows
that for every simulation setting, LAGR selection and estimation is
more accurate than the standard VCR model.

\begin{kframe}


{\ttfamily\noindent\color{warningcolor}{\#\# Warning: cannot open compressed file '/Users/wesley/scratch/gwr-sim-output.RData', probable reason 'No such file or directory'}}

{\ttfamily\noindent\bfseries\color{errorcolor}{\#\# Error: cannot open the connection}}

{\ttfamily\noindent\bfseries\color{errorcolor}{\#\# Error: object 'pzero' not found}}\end{kframe}

From Table \ref{tab:pzero} we see that LAGR has good ability to identify
non-important predictors. The frequency with which $\beta_{2}(\bm{s}),\dots,\beta_{5}(\bm{s})$
were dropped from the LAGR models ranged from \ensuremath{\infty{}}
to \ensuremath{-\infty{}}.

\begin{table}
	\begin{tabular}{cccccc}
		$\beta_{1}(\bm{s})$ & $\rho$ & $\sigma_{\varepsilon}^{2}$ & Frequency & LAGR & VCR & Oracle \\
		\hline 

\begin{kframe}


{\ttfamily\noindent\color{warningcolor}{\#\# Warning: cannot open compressed file '/Users/wesley/scratch/gwr-sim-output.RData', probable reason 'No such file or directory'}}

{\ttfamily\noindent\bfseries\color{errorcolor}{\#\# Error: cannot open the connection}}

{\ttfamily\noindent\bfseries\color{errorcolor}{\#\# Error: object 'misey' not found}}

{\ttfamily\noindent\bfseries\color{errorcolor}{\#\# Error: object 'misey' not found}}

{\ttfamily\noindent\bfseries\color{errorcolor}{\#\# Error: object 'misey' not found}}

{\ttfamily\noindent\bfseries\color{errorcolor}{\#\# Error: object 'misey' not found}}

{\ttfamily\noindent\bfseries\color{errorcolor}{\#\# Error: object 'misey' not found}}

{\ttfamily\noindent\bfseries\color{errorcolor}{\#\# Error: object 'misey.table' not found}}\end{kframe}
	\end{tabular}
	\caption{The MISE for the fitted output in each simulation setting, under variable selection via LAGR, no variable selection, and oracular variable selection. Highlighting indicates the \\textbf{closest} and \\emph{next-closest} to the actual error variance $\\sigma_\\varepsilon^2$ for that setting.}
	\label{tab:misey}
\end{table}

The MISE of the fitted $\hat{y}(\bm{s})$ is listed in Table \ref{tab:misey},
where the highlighting is based on which methods estimate an error
variance that is closest to the known truth for the simulation. The
results are all very similar to each other, indicating that no method
was consistently better than the others in this simulation at fitting
the model output.


\subsection{Discussion}

The proposed LAGR method was accurate in selection and estimation,
with estimation accuracy for $\beta_{1}(\bm{s})$ about equal to that
of the VCR model with no selection, and with consistently better accuracy
for estimating $\beta_{2}(\bm{s}),\dots,\beta_{5}(\bm{s})$.

There was minimal difference in the performance of the proposed LAGR
method between low ($\sigma_{\varepsilon}=0.5$) and high ($\sigma_{\varepsilon}=1$)
error variance, and between no ($\rho=0$) and moderate ($\rho=0.5$)
correlation among the predictor variables. But the selection and estimation
accuracy did decline when there was high ($\rho=0.9$) correlation
among the predictor variables.


\section{Data example\label{sec:example}}

The proposed LAGR estimation method was used to estimate the coefficients
in a VCR model of the effect of some covariates on the price of homes
in Boston. The data source is the Boston house price data set of \cite{Harrison-Rubinfeld-1978,Gilley-Pace-1996,Pace-Gilley-1997},
which is based on the 1970 U.S. census. In the data, we have the median
price of homes sold in 506 census tracts (MEDV), along with some potential
predictor variables. The predictor variables are CRIM (the per-capita
crime rate in the tract), RM (the mean number of rooms for houses
sold in the tract), RAD (an index of how accessible the tract is from
Boston's radial roads), TAX (the property tax per \$10,000 of property
value), and LSTAT (the percentage of the tract's residents who are
considered ``lower status'').

The bandwidth parameter was set to 0.2 for a nearest neighbors-type
bandwidth, meaning that the sum of kernel weights for each local model
was 20\% of the total number of observations. The kernel used was
the Epanechnikov kernel.


\subsection{Results}

Estimates of the regression coefficients are plotted in Figure \ref{fig:boston-lagr-coefs}.

\begin{figure}

\includegraphics[width=\maxwidth]{figure/boston-plots} 


\caption{The coefficients for the boston house price data as estimated by LAGR.\label{fig:boston-lagr-coefs}}
\end{figure}

One interesting result is that LAGR indicates that the TAX variable
was nowhere an important predictor of the median house price. Another
is that the coefficients of CRIM and LSTAT are everywhere negative
or zero (meaning that the increasing the crime rate or proportion
of lower-status individuals reduces the median house price where the
effect is discernable) and that of RM is positive (meaning that when
the average house in a tract has more rooms, the median house will
be more expensive), but the coefficient of RAD is positive in some
areas and negative in others. This indicates that there are parts
of Boston where improved access to radial roads increases the median
house price and parts where it decreases the median house price.

There is not an obvious spatial pattern to the local coefficients
for RAD - there are more tracts with negative coefficients than positive,
and the positive coefficients do appear to be clustered, but the tracts
with positive coefficients are also adjacent to tracts with negative
coefficients. Indeed, there is not an obvious spatial pattern to any
of the coefficient surfaces except for TAX, which is zero everywhere.

% latex table generated in R 3.1.0 by xtable 1.7-3 package
% Mon Jul  7 18:26:55 2014
\begin{table}
\centering
\begin{tabular}{rrrr}
  & Mean & SD & Prop. zero \\ 
  \hline
CRIM & -0.07 & 0.08 & 0.49 \\ 
  RM & 1.92 & 1.43 & 0.02 \\ 
  RAD & -0.08 & 0.13 & 0.37 \\ 
  TAX & 0.00 & 0.00 & 1.00 \\ 
  LSTAT & -0.72 & 0.16 & 0.01 \\ 
  \end{tabular}
\caption{The mean, standard deviation, and proportion of zeros among the local coefficients in a model for the median house price in census tracts in Boston, with coefficients selected and fitted by LAGR.} 
\label{tab:boston-coefs-lagr}
\end{table}


A summary of the local coefficients is in Table \ref{tab:boston-coefs-lagr}.
It indicates that RM is the only predictor variable with a positive
mean of the local coefficients, but also that the mean of the local
coefficients of RM is the largest coefficient - at 1.92,
it is more than twice as large in magnitude as the mean local coefficient
of LSTAT (\ensuremath{-0.72}),
which is second-largest.

The coefficient of the CRIM variable was estimated to be exactly zero
at 49\%
of the locations. The percentage for the RAD variable was 37\%.

In their example using the same data, \cite{Sun-Yan-Zhang-Lu-2014}
estimated that the coefficients of RAD annd LSTAT should be constant,
at 0.36 and -0.45, respectively. That conclusion differs from our
result, which says that the mean local coefficient of RAD is actually
negative (\ensuremath{-0.08}), while
our mean fitted local coefficient for LSTAT was more negative than
the estimate of \cite{Sun-Yan-Zhang-Lu-2014}.

\appendix

\section{Proofs of theorems\label{app:proofs} }
\begin{proof}[Proof of theorem \ref{theorem:normality}]

\end{proof}
Define $V_{4}^{(n)}(\bm{u})$ to be the 
\begin{align}
\mkern-36muV_{4}^{(n)}(\bm{u}) & =\mathcal{J}\left\{ \bm{\zeta}(\bm{s})+h^{-1}n^{-1/2}\bm{u}\right\} -\mathcal{J}\left\{ \bm{\zeta}(\bm{s})\right\} \notag\label{eq:consistency}\\
 & \mkern-36mu=(1/2)\left[\bm{Y}-\bm{Z}(\bm{s})\left\{ \bm{\zeta}(\bm{s})+h^{-1}n^{-1/2}\bm{u}\right\} \right]^{T}\bm{W}(\bm{s})\left[\bm{Y}-\bm{Z}(\bm{s})\left\{ \bm{\zeta}(\bm{s})+h^{-1}n^{-1/2}\bm{u}\right\} \right]\notag\\
 & +\sum_{j=1}^{p}\phi_{j}(\bm{s})\|\bm{\zeta}_{j}(\bm{s})+h^{-1}n^{-1/2}\bm{u}_{j}\|\notag\\
 & -(1/2)\left\{ \bm{Y}-\bm{Z}(\bm{s})\bm{\zeta}(\bm{s})\right\} ^{T}\bm{W}(\bm{s})\left\{ \bm{Y}-\bm{Z}(\bm{s})\bm{\zeta}(\bm{s})\right\} -\sum_{j=1}^{p}\phi_{j}(\bm{s})\|\bm{\zeta}_{j}(\bm{s})\|\notag\\
 & \mkern-36mu=(1/2)\bm{u}^{T}\left\{ h^{-2}n^{-1}\bm{Z}^{T}(\bm{s})\bm{W}(\bm{s})\bm{Z}(\bm{s})\right\} \bm{u}-\bm{u}^{T}\left[h^{-1}n^{-1/2}\bm{Z}^{T}(\bm{s})\bm{W}(\bm{s})\left\{ \bm{Y}-\bm{Z}(\bm{s})\bm{\zeta}(\bm{s})\right\} \right]\notag\\
 & +\sum_{j=1}^{p}n^{-1/2}\phi_{j}(\bm{s})n^{1/2}\left\{ \|\bm{\zeta}_{j}(\bm{s})+h^{-1}n^{-1/2}\bm{u}_{j}\|-\|\bm{\zeta}_{j}(\bm{s})\|\right\} 
\end{align}


Note the different limiting behavior of the third term between the
cases $j\le p_{0}$ and $j>p_{0}$:


\paragraph{Case $j\le p_{0}$}

If $j\le p_{0}$ then $n^{-1/2}\phi_{j}(\bm{s})\to n^{-1/2}\lambda_{n}(\bm{s})\|\bm{\zeta}_{j}(\bm{s})\|^{-\gamma}$
and $|\sqrt{n}\left\{ \|\bm{\zeta}_{j}(\bm{s})+h^{-1}n^{-1/2}\bm{u}_{j}\|-\|\bm{\zeta}_{j}(\bm{s})\|\right\} |\le h^{-1}\|\bm{u}_{j}\|$
so 
\[
\lim\limits _{n\to\infty}\phi_{j}(\bm{s})\left(\|\bm{\zeta}_{j}(\bm{s})+h^{-1}n^{-1/2}\bm{u}_{j}\|-\|\bm{\zeta}_{j}(\bm{s})\|\right)\le h^{-1}n^{-1/2}\phi_{j}(\bm{s})\|\bm{u}_{j}\|\le h^{-1}n^{-1/2}a_{n}\|\bm{u}_{j}\|\to0
\]



\paragraph{Case $j>p_{0}$}

If $j>p_{0}$ then $\phi_{j}(\bm{s})\left(\|\bm{\zeta}_{j}(\bm{s})+h^{-1}n^{-1/2}\bm{u}_{j}\|-\|\bm{\zeta}_{j}(\bm{s})\|\right)=\phi_{j}(\bm{s})h^{-1}n^{-1/2}\|\bm{u}_{j}\|$.

And note that $h=O(n^{-1/6})$ so that if $hn^{-1/2}b_{n}\xrightarrow{p}\infty$
then $h^{-1}n^{-1/2}b_{n}\xrightarrow{p}\infty$.

Now, if $\|\bm{u}_{j}\|\ne0$ then 
\[
h^{-1}n^{-1/2}\phi_{j}(\bm{s})\|\bm{u}_{j}\|\ge h^{-1}n^{-1/2}b_{n}\|\bm{u}_{j}\|\to\infty
\]
. On the other hand, if $\|\bm{u}_{j}\|=0$ then $h^{-1}n^{-1/2}\phi_{j}(\bm{s})\|\bm{u}_{j}\|=0$.

Thus, the limit of $V_{4}^{(n)}(\bm{u})$ is the same as the limit
of $V_{4}^{*(n)}(\bm{u})$ where

\[
\mkern-72muV_{4}^{*(n)}(\bm{u})=\begin{cases}
(1/2)\bm{u}^{T}\left\{ h^{-2}n^{-1}\bm{Z}^{T}(\bm{s})\bm{W}(\bm{s})\bm{Z}(\bm{s})\right\} \bm{u}-\bm{u}^{T}\left[h^{-1}n^{-1/2}\bm{Z}^{T}(\bm{s})\bm{W}(\bm{s})\left\{ \bm{Y}-\bm{Z}(\bm{s})\bm{\zeta}(\bm{s})\right\} \right] & \mbox{ if }\|\bm{u}_{j}\|=0\;\forall j>p_{0}\\
\infty & \mbox{ otherwise }
\end{cases}.
\]


From which it is clear that $V_{4}^{*(n)}(\bm{u})$ is convex and
its unique minimizer is $\hat{\bm{u}}^{(n)}$:

\begin{align}
0 & =\left\{ h^{-2}n^{-1}\bm{Z}^{T}(\bm{s})\bm{W}(\bm{s})\bm{Z}(\bm{s})\right\} \hat{\bm{u}}^{(n)}-\left[h^{-1}n^{-1/2}\bm{Z}^{T}(\bm{s})\bm{W}(\bm{s})\left\{ \bm{Y}-\bm{Z}(\bm{s})\bm{\zeta}(\bm{s})\right\} \right]\notag\label{eq:limit}\\
\therefore\hat{\bm{u}}^{(n)} & =\left\{ n^{-1}\bm{Z}^{T}(\bm{s})\bm{W}(\bm{s})\bm{Z}(\bm{s})\right\} ^{-1}\left[hn^{-1/2}\bm{Z}^{T}(\bm{s})\bm{W}(\bm{s})\left\{ \bm{Y}-\bm{Z}(\bm{s})\bm{\zeta}(\bm{s})\right\} \right]\notag\\
\end{align}


By the epiconvergence results of \cite{Geyer-1994} and \cite{Knight-Fu-2000},
the minimizer of the limiting function is the limit of the minimizers
$\hat{\bm{u}}^{(n)}$. And since, by Lemma 2 of \cite{Sun-Yan-Zhang-Lu-2014},

\begin{equation}
\hat{\bm{u}}^{(n)}\xrightarrow{d}N\left(\frac{\kappa_{2}h^{2}}{2\kappa_{0}}\{\nabla_{uu}^{2}\bm{\zeta}_{j}(\bm{s})+\nabla_{vv}^{2}\bm{\zeta}_{j}(\bm{s})\},f(\bm{s})\kappa_{0}^{-2}\nu_{0}\sigma^{2}\Psi^{-1}\right)
\end{equation}
the result is proven.
\begin{proof}[Proof of theorem \ref{theorem:selection}]


We showed in Theorem \ref{theorem:normality} that $\hat{\bm{\zeta}}_{j}(\bm{s})\xrightarrow{p}\bm{\zeta}_{j}(\bm{s})+\frac{\kappa_{2}h^{2}}{2\kappa_{0}}\{\nabla_{uu}^{2}\bm{\zeta}_{j}(\bm{s})+\nabla_{vv}^{2}\bm{\zeta}_{j}(\bm{s})\}$,
so to complete the proof of selection consistency, it only remains
to show that $P\left\{ \hat{\bm{\zeta}}_{j}(\bm{s})=0\right\} \to1$
if $j>p_{0}$.
\end{proof}
The proof is by contradiction. Without loss of generality we consider
only the case $j=p$.

Assume $\|\hat{\bm{\zeta}}_{p}(\bm{s})\|\ne0$. Then $Q\left\{ \bm{\zeta}(\bm{s})\right\} $
is differentiable w.r.t. $\bm{\zeta}_{p}(\bm{s})$ and is minimized
where 
\begin{align}
0 & =\bm{Z}_{p}^{T}(\bm{s})\bm{W}(\bm{s})\left\{ \bm{Y}-\bm{Z}_{-p}(\bm{s})\hat{\bm{\zeta}}_{-p}(\bm{s})-\bm{Z}_{p}(\bm{s})\hat{\bm{\zeta}}_{p}(\bm{s})\right\} -\phi_{p}(\bm{s})\frac{\hat{\bm{\zeta}}_{p}(\bm{s})}{\|\hat{\bm{\zeta}}_{p}(\bm{s})\|}\notag\\
 & =\bm{Z}_{p}^{T}(\bm{s})\bm{W}(\bm{s})\left[\bm{Y}-\bm{Z}(\bm{s})\bm{\zeta}(\bm{s})-\frac{h^{2}\kappa_{2}}{2\kappa_{0}}\left\{ \nabla_{uu}^{2}\bm{\zeta}(\bm{s})+\nabla_{vv}^{2}\bm{\zeta}(\bm{s})\right\} \right]\notag\\
 & \mkern+72mu+\bm{Z}_{p}^{T}(\bm{s})\bm{W}(\bm{s})\bm{Z}_{-p}(\bm{s})\left[\bm{\zeta}_{-p}(\bm{s})+\frac{h^{2}\kappa_{2}}{2\kappa_{0}}\left\{ \nabla_{uu}^{2}\bm{\zeta}_{-p}(\bm{s})+\nabla_{vv}^{2}\bm{\zeta}_{-p}(\bm{s})\right\} -\hat{\bm{\zeta}}_{-p}(\bm{s})\right]\notag\\
 & \mkern+72mu+\bm{Z}_{p}^{T}(\bm{s})\bm{W}(\bm{s})\bm{Z}_{p}(\bm{s})\left[\bm{\zeta}_{p}(\bm{s})+\frac{h^{2}\kappa_{2}}{2\kappa_{0}}\left\{ \nabla_{uu}^{2}\bm{\zeta}_{p}(\bm{s})+\nabla_{vv}^{2}\bm{\zeta}_{p}(\bm{s})\right\} -\hat{\bm{\zeta}}_{p}(\bm{s})\right]\notag\\
 & \mkern+72mu-\phi_{p}(\bm{s})\frac{\hat{\bm{\zeta}}_{p}(\bm{s})}{\|\hat{\bm{\zeta}}_{p}(\bm{s})\|}\notag\\
\end{align}


So 
\begin{align}
\frac{h}{\sqrt{n}}\phi_{p}(\bm{s})\frac{\hat{\bm{\zeta}}_{p}(\bm{s})}{\|\hat{\bm{\zeta}}_{p}(\bm{s})\|} & =\bm{Z}_{p}^{T}(\bm{s})\bm{W}(\bm{s})\frac{h}{\sqrt{n}}\left[\bm{Y}-\bm{Z}(\bm{s})\bm{\zeta}(\bm{s})-\frac{h^{2}\kappa_{2}}{2\kappa_{0}}\left\{ \nabla_{uu}^{2}\bm{\zeta}(\bm{s})+\nabla_{vv}^{2}\bm{\zeta}(\bm{s})\right\} \right]\notag\label{eq:selection}\\
 & +\left\{ n^{-1}\bm{Z}_{p}^{T}(\bm{s})\bm{W}(\bm{s})\bm{Z}_{-p}(\bm{s})\right\} h\sqrt{n}\left[\bm{\zeta}_{-p}(\bm{s})+\frac{h^{2}\kappa_{2}}{2\kappa_{0}}\left\{ \nabla_{uu}^{2}\bm{\zeta}_{-p}(\bm{s})+\nabla_{vv}^{2}\bm{\zeta}_{-p}(\bm{s})\right\} -\hat{\bm{\zeta}}_{-p}(\bm{s})\right]\notag\\
 & +\left\{ n^{-1}\bm{Z}_{p}^{T}(\bm{s})\bm{W}(\bm{s})\bm{Z}_{p}(\bm{s})\right\} h\sqrt{n}\left[\bm{\zeta}_{p}(\bm{s})+\frac{h^{2}\kappa_{2}}{2\kappa_{0}}\left\{ \nabla_{uu}^{2}\bm{\zeta}_{p}(\bm{s})+\nabla_{vv}^{2}\bm{\zeta}_{p}(\bm{s})\right\} -\hat{\bm{\zeta}}_{p}(\bm{s})\right]
\end{align}


From Lemma 2 of \cite{Sun-Yan-Zhang-Lu-2014}, $\left\{ n^{-1}\bm{Z}_{p}^{T}(\bm{s})\bm{W}(\bm{s})\bm{Z}_{-p}(\bm{s})\right\} =O_{p}(1)$
and $\left\{ n^{-1}\bm{Z}_{p}^{T}(\bm{s})\bm{W}(\bm{s})\bm{Z}_{p}(\bm{s})\right\} =O_{p}(1)$.

From Theorem 3 of \cite{Sun-Yan-Zhang-Lu-2014}, we have that $h\sqrt{n}\left[\hat{\bm{\zeta}}_{-p}(\bm{s})-\bm{\zeta}_{-p}(\bm{s})-\frac{h^{2}\kappa_{2}}{2\kappa_{0}}\left\{ \nabla_{uu}^{2}\zeta_{-p}(\bm{s})+\nabla_{vv}^{2}\zeta_{-p}(\bm{s})\right\} \right]=O_{p}(1)$
and $h\sqrt{n}\left[\hat{\bm{\zeta}}_{p}(\bm{s})-\bm{\zeta}_{p}(\bm{s})-\frac{h^{2}\kappa_{2}}{2\kappa_{0}}\left\{ \nabla_{uu}^{2}\zeta_{p}(\bm{s})+\nabla_{vv}^{2}\zeta_{p}(\bm{s})\right\} \right]=O_{p}(1)$.

So the second and third terms of the sum in (\ref{eq:selection})
are $O_{p}(1)$.

We showed in the proof of \ref{theorem:normality} that $h\sqrt{n}\bm{Z}_{p}^{T}(\bm{s})\bm{W}(\bm{s})\left[\bm{Y}-\bm{Z}(\bm{s})\bm{\zeta}(\bm{s})-\frac{h^{2}\kappa_{2}}{2\kappa_{0}}\left\{ \nabla_{uu}^{2}\bm{\zeta}(\bm{s})+\nabla_{vv}^{2}\bm{\zeta}(\bm{s})\right\} \right]=O_{p}(1)$.

The three terms of the sum to the right of the equals sign in (\ref{eq:selection})
are $O_{p}(1)$, so for $\hat{\bm{\zeta}}_{p}(\bm{s})$ to be a solution,
we must have that $hn^{-1/2}\phi_{p}(\bm{s})\hat{\bm{\zeta}}_{p}(\bm{s})/\|\hat{\bm{\zeta}}_{p}(\bm{s})\|=O_{p}(1)$.

But since by assumption $\hat{\bm{\zeta}}_{p}(\bm{s})\ne0$, there
must be some $k\in\{1,\dots,3\}$ such that $|\hat{\zeta}_{p_{k}}(\bm{s})|=\max\{|\hat{\zeta}_{p_{k'}}(\bm{s})|:1\le k'\le3\}$.
And for this $k$, we have that $|\hat{\zeta}_{p_{k}}(\bm{s})|/\|\hat{\bm{\zeta}}_{p}(\bm{s})\|\ge1/\sqrt{3}>0$.

Now since $hn^{-1/2}b_{n}\to\infty$, we have that $hn^{-1/2}\phi_{p}(\bm{s})\hat{\bm{\zeta}}_{p}(\bm{s})/\|\hat{\bm{\zeta}}_{p}(\bm{s})\|\ge hb_{n}/\sqrt{3n}\to\infty$
and therefore the term to the left of the equals sign dominates the
sum to the right of the equals sign in (\ref{eq:selection}). So for
large enough $n$, $\hat{\bm{\zeta}}_{p}(\bm{s})\ne0$ cannot maximize
$Q$.

So $P\left\{ \hat{\bm{\zeta}}_{(b)}(\bm{s})=0\right\} \to1$. 

\begin{comment}
\begin{proof}[Proof of theorem \ref{theorem:normality-glm}]


Define the $q$-functions to be the derivatives of the quasi-likelihood:
$q_{j}(t,y)=\left(\partial/\partial t\right)^{j}Q\left\{ g^{-1}\left(t\right),y\right\} $.
Then $q_{1}\left\{ \eta\left(\bm{s}\right),\mu\left(\bm{s}\right)\right\} =0$
and $q_{2}\left\{ \eta\left(\bm{s}\right),\mu\left(\bm{s}\right)\right\} =-b''\left\{ \eta\left(\bm{s}\right)\right\} $.
Also define $\bar{\eta}_{i}\left(\right)$
\end{proof}
Define $H^{(n)}(\bm{u})$ to be 
\begin{align}
H^{(n)}(\bm{u})= & \mathcal{J}^{*}\left\{ \bm{\zeta}(\bm{s})+h^{-1}n^{-1/2}\bm{u}\right\} -\mathcal{J}^{*}\left\{ \bm{\zeta}(\bm{s})\right\} \notag\label{eq:consistency-glm}\\
= & \sum_{i=1}^{n}K_{h}(\|\bm{s}-\bm{s}_{i}\|)Q\left[g^{-1}\left\{ \bm{z}'(\bm{s}_{i})\bm{\zeta}(\bm{s})+h^{-1}n^{-1/2}\bm{u}\right\} ,Y(\bm{s}_{i})\right]\\
 & -\sum_{i=1}^{n}K_{h}(\|\bm{s}-\bm{s}_{i}\|)Q\left[g^{-1}\left\{ \bm{z}'(\bm{s}_{i})\bm{\zeta}(\bm{s})\right\} ,Y(\bm{s}_{i})\right]\\
 & +\sum_{j=1}^{p}\phi_{j}(\bm{s})\|\bm{\zeta}_{j}(\bm{s})+h^{-1}n^{-1/2}\bm{u}\|-+\sum_{j=1}^{p}\phi_{j}(\bm{s})\|\bm{\zeta}_{j}(\bm{s})\|
\end{align}


Letting $A_{1}^{T}\left(\bm{s}\right)=\sum_{i=1}^{n}q_{1}$ taking
the Taylor expansion of $\mathcal{J}^{*}\left\{ \bm{\zeta}(\bm{s})+h^{-1}n^{-1/2}\bm{u}\right\} $
around $\bm{\zeta}\left(\bm{s}\right)$ give us:

\begin{align}
\mathcal{J}^{*}\left\{ \bm{\zeta}(\bm{s})+h^{-1}n^{-1/2}\bm{u}\right\} = & \mathcal{J}^{*}\left\{ \bm{\zeta}(\bm{s})\right\} +A_{1}^{T}\left(\bm{s}\right)\bm{u}+(1/2)\bm{u}'A_{2}^{T}\left(\bm{s}\right)\bm{u}+\left(h^{-3}n^{-3/2}/6\right)\sum_{i=1}^{n}K_{h}\left(\|\bm{s}-\bm{s}_{i}\|\right)q_{3}\left[\left\{ \bm{Z}(\bm{s}_{i})\right\} _{i}^{T}\tilde{\bm{\zeta}}_{i},Y\left(\bm{s}_{i}\right)\right]\left[\left\{ \bm{Z}(\bm{s}_{i})\right\} _{i}^{T}\bm{u}\right]^{3}\label{eq:glm-Taylor-expansion}\\
 & +\sum_{j=1}^{p}\phi_{j}(\bm{s})\|\bm{\zeta}_{j}(\bm{s})+h^{-1}n^{-1/2}\bm{u}\|\nonumber 
\end{align}


where $\tilde{\bm{\zeta}_{i}}$ lies between $\bm{\zeta}(\bm{s})$
and $\bm{\zeta}(\bm{s})+h^{-1}n^{-1/2}\bm{u}$. Now the difference
$H^{(n)}(\bm{u})=\mathcal{J}^{*}\left\{ \bm{\zeta}(\bm{s})+h^{-1}n^{-1/2}\bm{u}\right\} -\mathcal{J}^{*}\left\{ \bm{\zeta}(\bm{s})\right\} $
is 

\begin{align*}
H^{(n)}(\bm{u})= & h^{-1}n^{-1/2}\bm{u}\sum_{i=1}^{n}q_{1}\left\{ \bm{Z}'(\bm{s}_{i})\bm{\zeta}(\bm{s}),Y\left(\bm{s}_{i}\right)\right\} K_{h}\left(\|\bm{s}-\bm{s}_{i}\|\right)\\
 & +h^{-2}n^{-1}\bm{u}'\bm{u}\sum_{i=1}^{n}q_{2}\left\{ \bm{Z}'(\bm{s}_{i})\bm{\zeta}(\bm{s}),Y\left(\bm{s}_{i}\right)\right\} K_{h}\left(\|\bm{s}-\bm{s}_{i}\|\right)\\
 & +\left(h^{-1}n^{-1/2}\bm{u}\right)^{3}\sum_{i=1}^{n}q_{3}\left\{ \tilde{\eta}_{i},Y\left(\bm{s}_{i}\right)\right\} K_{h}\left(\|\bm{s}-\bm{s}_{i}\|\right)\\
 & +\sum_{j=1}^{p}\phi_{j}(\bm{s})\left\{ \|\bm{\zeta}_{j}(\bm{s})+h^{-1}n^{-1/2}\bm{u}\|-\|\bm{\zeta}_{j}(\bm{s})\|\right\} 
\end{align*}


Note the different limiting behavior of the third term between the
cases $j\le p_{0}$ and $j>p_{0}$:


\paragraph{Case $j\le p_{0}$}

If $j\le p_{0}$ then $n^{-1/2}\phi_{j}(\bm{s})\to n^{-1/2}\lambda_{n}(\bm{s})\|\bm{\zeta}_{j}(\bm{s})\|^{-\gamma}$
and $|\sqrt{n}\left\{ \|\bm{\zeta}_{j}(\bm{s})+h^{-1}n^{-1/2}\bm{u}_{j}\|-\|\bm{\zeta}_{j}(\bm{s})\|\right\} |\le h^{-1}\|\bm{u}_{j}\|$
so 
\[
\lim\limits _{n\to\infty}\phi_{j}(\bm{s})\left(\|\bm{\zeta}_{j}(\bm{s})+h^{-1}n^{-1/2}\bm{u}_{j}\|-\|\bm{\zeta}_{j}(\bm{s})\|\right)\le h^{-1}n^{-1/2}\phi_{j}(\bm{s})\|\bm{u}_{j}\|\le h^{-1}n^{-1/2}a_{n}\|\bm{u}_{j}\|\to0
\]



\paragraph{Case $j>p_{0}$}

If $j>p_{0}$ then $\phi_{j}(\bm{s})\left(\|\bm{\zeta}_{j}(\bm{s})+h^{-1}n^{-1/2}\bm{u}_{j}\|-\|\bm{\zeta}_{j}(\bm{s})\|\right)=\phi_{j}(\bm{s})h^{-1}n^{-1/2}\|\bm{u}_{j}\|$.

And note that $h=O(n^{-1/6})$ so that if $hn^{-1/2}b_{n}\xrightarrow{p}\infty$
then $h^{-1}n^{-1/2}b_{n}\xrightarrow{p}\infty$.

Now, if $\|\bm{u}_{j}\|\ne0$ then 
\[
h^{-1}n^{-1/2}\phi_{j}(\bm{s})\|\bm{u}_{j}\|\ge h^{-1}n^{-1/2}b_{n}\|\bm{u}_{j}\|\to\infty
\]
. On the other hand, if $\|\bm{u}_{j}\|=0$ then $h^{-1}n^{-1/2}\phi_{j}(\bm{s})\|\bm{u}_{j}\|=0$.

Thus, the limit of $V_{4}^{(n)}(\bm{u})$ is the same as the limit
of $V_{4}^{*(n)}(\bm{u})$ where

\[
\mkern-72muV_{4}^{*(n)}(\bm{u})=\begin{cases}
(1/2)\bm{u}^{T}\left\{ h^{-2}n^{-1}\bm{Z}^{T}(\bm{s})\bm{T}(\bm{s})\bm{Z}(\bm{s})\right\} \bm{u}-\bm{u}^{T}\left[h^{-1}n^{-1/2}\bm{Z}^{T}(\bm{s})\bm{T}(\bm{s})\left\{ \bm{\Omega}-\bm{Z}(\bm{s})\bm{\zeta}(\bm{s})\right\} \right] & \mbox{ if }\|\bm{u}_{j}\|=0\;\forall j>p_{0}\\
\infty & \mbox{ otherwise }
\end{cases}.
\]


From which it is clear that $V_{4}^{*(n)}(\bm{u})$ is convex and
its unique minimizer is $\hat{\bm{u}}^{(n)}$:

\begin{align}
0 & =\left\{ h^{-2}n^{-1}\bm{Z}^{T}(\bm{s})\bm{T}(\bm{s})\bm{Z}(\bm{s})\right\} \hat{\bm{u}}^{(n)}-\left[h^{-1}n^{-1/2}\bm{Z}^{T}(\bm{s})\bm{T}(\bm{s})\left\{ \bm{\Omega}-\bm{Z}(\bm{s})\bm{\zeta}(\bm{s})\right\} \right]\notag\label{eq:limit-glm}\\
\therefore\hat{\bm{u}}^{(n)} & =\left\{ n^{-1}\bm{Z}^{T}(\bm{s})\bm{T}(\bm{s})\bm{Z}(\bm{s})\right\} ^{-1}\left[hn^{-1/2}\bm{Z}^{T}(\bm{s})\bm{T}(\bm{s})\left\{ \bm{\Omega}-\bm{Z}(\bm{s})\bm{\zeta}(\bm{s})\right\} \right]\notag\\
\end{align}


By the epiconvergence results of \cite{Geyer-1994} and \cite{Knight-Fu-2000},
the minimizer of the limiting function is the limit of the minimizers
$\hat{\bm{u}}^{(n)}$. And if

\begin{equation}
\hat{\bm{u}}^{(n)}\xrightarrow{d}N\left(\frac{\kappa_{2}h^{2}}{2\kappa_{0}}\{\nabla_{uu}^{2}\bm{\zeta}_{j}(\bm{s})+\nabla_{vv}^{2}\bm{\zeta}_{j}(\bm{s})\},f(\bm{s})\kappa_{0}^{-2}\nu_{0}\sigma^{2}\Gamma\left(\bm{s}\right)^{-1}\right)
\end{equation}
the result is proven.
\begin{proof}[Proof of theorem \ref{theorem:selection-glm}]


We showed in Theorem \ref{theorem:normality-glm} that $\hat{\bm{\zeta}}_{j}(\bm{s})\xrightarrow{p}\bm{\zeta}_{j}(\bm{s})+\frac{\kappa_{2}h^{2}}{2\kappa_{0}}\{\nabla_{uu}^{2}\bm{\zeta}_{j}(\bm{s})+\nabla_{vv}^{2}\bm{\zeta}_{j}(\bm{s})\}$,
so to complete the proof of selection consistency, it only remains
to show that $P\left\{ \hat{\bm{\zeta}}_{j}(\bm{s})=0\right\} \to1$
if $j>p_{0}$.
\end{proof}
The proof is by contradiction. Without loss of generality we consider
only the case $j=p$.

Assume $\|\hat{\bm{\zeta}}_{p}(\bm{s})\|\ne0$. Then $Q\left\{ \bm{\zeta}(\bm{s})\right\} $
is differentiable w.r.t. $\bm{\zeta}_{p}(\bm{s})$ and is minimized
where 
\begin{align}
0 & =\bm{Z}_{p}^{T}(\bm{s})\bm{T}(\bm{s})\left\{ \bm{\Omega}-\bm{Z}_{-p}(\bm{s})\hat{\bm{\zeta}}_{-p}(\bm{s})-\bm{Z}_{p}(\bm{s})\hat{\bm{\zeta}}_{p}(\bm{s})\right\} -\phi_{p}(\bm{s})\frac{\hat{\bm{\zeta}}_{p}(\bm{s})}{\|\hat{\bm{\zeta}}_{p}(\bm{s})\|}\notag\\
 & =\bm{Z}_{p}^{T}(\bm{s})T(\bm{s})\left[\bm{\Omega}-\bm{Z}(\bm{s})\bm{\zeta}(\bm{s})-\frac{h^{2}\kappa_{2}}{2\kappa_{0}}\left\{ \nabla_{uu}^{2}\bm{\zeta}(\bm{s})+\nabla_{vv}^{2}\bm{\zeta}(\bm{s})\right\} \right]\notag\\
 & \mkern+72mu+\bm{Z}_{p}^{T}(\bm{s})\bm{T}(\bm{s})\bm{Z}_{-p}(\bm{s})\left[\bm{\zeta}_{-p}(\bm{s})+\frac{h^{2}\kappa_{2}}{2\kappa_{0}}\left\{ \nabla_{uu}^{2}\bm{\zeta}_{-p}(\bm{s})+\nabla_{vv}^{2}\bm{\zeta}_{-p}(\bm{s})\right\} -\hat{\bm{\zeta}}_{-p}(\bm{s})\right]\notag\\
 & \mkern+72mu+\bm{Z}_{p}^{T}(\bm{s})\bm{T}(\bm{s})\bm{Z}_{p}(\bm{s})\left[\bm{\zeta}_{p}(\bm{s})+\frac{h^{2}\kappa_{2}}{2\kappa_{0}}\left\{ \nabla_{uu}^{2}\bm{\zeta}_{p}(\bm{s})+\nabla_{vv}^{2}\bm{\zeta}_{p}(\bm{s})\right\} -\hat{\bm{\zeta}}_{p}(\bm{s})\right]\notag\\
 & \mkern+72mu-\phi_{p}(\bm{s})\frac{\hat{\bm{\zeta}}_{p}(\bm{s})}{\|\hat{\bm{\zeta}}_{p}(\bm{s})\|}\notag\\
\end{align}


So 
\begin{align}
\frac{h}{\sqrt{n}}\phi_{p}(\bm{s})\frac{\hat{\bm{\zeta}}_{p}(\bm{s})}{\|\hat{\bm{\zeta}}_{p}(\bm{s})\|} & =\bm{Z}_{p}^{T}(\bm{s})\bm{T}(\bm{s})\frac{h}{\sqrt{n}}\left[\bm{\Omega}-\bm{Z}(\bm{s})\bm{\zeta}(\bm{s})-\frac{h^{2}\kappa_{2}}{2\kappa_{0}}\left\{ \nabla_{uu}^{2}\bm{\zeta}(\bm{s})+\nabla_{vv}^{2}\bm{\zeta}(\bm{s})\right\} \right]\notag\label{eq:selection-glm}\\
 & +\left\{ n^{-1}\bm{Z}_{p}^{T}(\bm{s})\bm{T}(\bm{s})\bm{Z}_{-p}(\bm{s})\right\} h\sqrt{n}\left[\bm{\zeta}_{-p}(\bm{s})+\frac{h^{2}\kappa_{2}}{2\kappa_{0}}\left\{ \nabla_{uu}^{2}\bm{\zeta}_{-p}(\bm{s})+\nabla_{vv}^{2}\bm{\zeta}_{-p}(\bm{s})\right\} -\hat{\bm{\zeta}}_{-p}(\bm{s})\right]\notag\\
 & +\left\{ n^{-1}\bm{Z}_{p}^{T}(\bm{s})\bm{T}(\bm{s})\bm{Z}_{p}(\bm{s})\right\} h\sqrt{n}\left[\bm{\zeta}_{p}(\bm{s})+\frac{h^{2}\kappa_{2}}{2\kappa_{0}}\left\{ \nabla_{uu}^{2}\bm{\zeta}_{p}(\bm{s})+\nabla_{vv}^{2}\bm{\zeta}_{p}(\bm{s})\right\} -\hat{\bm{\zeta}}_{p}(\bm{s})\right]
\end{align}


By assumption, $\left\{ n^{-1}\bm{Z}_{p}^{T}(\bm{s})\bm{T}(\bm{s})\bm{Z}_{-p}(\bm{s})\right\} =O_{p}(1)$
and $\left\{ n^{-1}\bm{Z}_{p}^{T}(\bm{s})\bm{T}(\bm{s})\bm{Z}_{p}(\bm{s})\right\} =O_{p}(1)$.

From Theorem 3 of \cite{Sun-Yan-Zhang-Lu-2014}, we have that $h\sqrt{n}\left[\hat{\bm{\zeta}}_{-p}(\bm{s})-\bm{\zeta}_{-p}(\bm{s})-\frac{h^{2}\kappa_{2}}{2\kappa_{0}}\left\{ \nabla_{uu}^{2}\zeta_{-p}(\bm{s})+\nabla_{vv}^{2}\zeta_{-p}(\bm{s})\right\} \right]=O_{p}(1)$
and $h\sqrt{n}\left[\hat{\bm{\zeta}}_{p}(\bm{s})-\bm{\zeta}_{p}(\bm{s})-\frac{h^{2}\kappa_{2}}{2\kappa_{0}}\left\{ \nabla_{uu}^{2}\zeta_{p}(\bm{s})+\nabla_{vv}^{2}\zeta_{p}(\bm{s})\right\} \right]=O_{p}(1)$.

So the second and third terms of the sum in (\ref{eq:selection-glm})
are $O_{p}(1)$.

We showed in the proof of \ref{theorem:normality} that $h\sqrt{n}\bm{Z}_{p}^{T}(\bm{s})\bm{T}(\bm{s})\left[\bm{\Omega}-\bm{Z}(\bm{s})\bm{\zeta}(\bm{s})-\frac{h^{2}\kappa_{2}}{2\kappa_{0}}\left\{ \nabla_{uu}^{2}\bm{\zeta}(\bm{s})+\nabla_{vv}^{2}\bm{\zeta}(\bm{s})\right\} \right]=O_{p}(1)$.

The three terms of the sum to the right of the equals sign in (\ref{eq:selection-glm})
are $O_{p}(1)$, so for $\hat{\bm{\zeta}}_{p}(\bm{s})$ to be a solution,
we must have that $hn^{-1/2}\phi_{p}(\bm{s})\hat{\bm{\zeta}}_{p}(\bm{s})/\|\hat{\bm{\zeta}}_{p}(\bm{s})\|=O_{p}(1)$.

But since by assumption $\hat{\bm{\zeta}}_{p}(\bm{s})\ne0$, there
must be some $k\in\{1,\dots,3\}$ such that $|\hat{\zeta}_{p_{k}}(\bm{s})|=\max\{|\hat{\zeta}_{p_{k'}}(\bm{s})|:1\le k'\le3\}$.
And for this $k$, we have that $|\hat{\zeta}_{p_{k}}(\bm{s})|/\|\hat{\bm{\zeta}}_{p}(\bm{s})\|\ge1/\sqrt{3}>0$.

Now since $hn^{-1/2}b_{n}\to\infty$, we have that $hn^{-1/2}\phi_{p}(\bm{s})\hat{\bm{\zeta}}_{p}(\bm{s})/\|\hat{\bm{\zeta}}_{p}(\bm{s})\|\ge hb_{n}/\sqrt{3n}\to\infty$
and therefore the term to the left of the equals sign dominates the
sum to the right of the equals sign in (\ref{eq:selection-glm}).
So for large enough $n$, $\hat{\bm{\zeta}}_{p}(\bm{s})\ne0$ cannot
maximize $Q$.

So $P\left\{ \hat{\bm{\zeta}}_{(b)}(\bm{s})=0\right\} \to1$. 
\end{comment}


\bibliographystyle{chicago}
\bibliography{3_Users_wesley_git_gwr_references_gwr}

\end{document}
